%%%%%%%%%%%%%%%%%%%%%%%%%%%%%%%%%%%%%%%%%%%%%%%%%%%%%%%%%%%%%%%%%%%%%%%%%%%%%%%%%%%%%%%%%%%%%%%%%%%%%%%%%%%%%%%%%%%%%%
	\chapter{ Cosmological Background }\label{chap:marco}
%%%%%%%%%%%%%%%%%%%%%%%%%%%%%%%%%%%%%%%%%%%%%%%%%%%%%%%%%%%%%%%%%%%%%%%%%%%%%%%%%%%%%%%%%%%%%%%%%%%%%%%%%%%%%%%%%%%%%%

Cosmology is the branch of physics that studies the Universe as a whole, 
therefore, it attempts to explain its origin, evolution and structure 
at big scales. Hence, a coarse grained approximation is mandatory due
to the scales considered, 
this is, several approximations are necessary in the endeavour of such a task.

In this search, two major points are considered. The first one is the 
cosmological principle, it assumes that on sufficiently large scales the Universe 
is homogeneous and isotropic. In this context, homogeneity can be understood like invariance 
under traslation and isotropy like invariance under rotation. Then, this principe establishes that
the universe should appear the same for certain observers named fundamental observers,
i.e., observers located at each point for which the universe appears isotropic and
homogeneous. Since the Universe is expanding, the distance among cosmological observers
changes with time but in an uniform way. Using these observers is possible to synchronize 
clocks using a light pulse. The time measured is named cosmic time. 


The overall isotropy and homogeneity have been observed, for example in observations of 
cosmic microwave background (CMB) radiation and the sponge like structure
of the distribution of galaxies. Until now, observations have agreed with this asseveration.

\

The second important point is that modern cosmology is fundamented on general relativity.
Here, Einstein field equations (EFE) serve as a set of fundamental equations to study the 
Universe at big scales. Fortunately, isotropy and homogeneity leds to a simple form of 
these and hence a relative simple mathematical treatment in cosmology.
From EFE, Friedmman equations are obtained, they provide a theoretical 
framework to study universe expansion. This is measured with the 
scale factor, i.e., describes how the relative distances between any two observers
change with cosmic time.
Other important issue is global curvature that
depends on the universe total content of energy and matter and can
take only three possible values, due to cosmological principle. 

A standard model in cosmology is $\lambda$CDM, where additionally to an 
expanding universe, there is a dark energy component that accelerates 
its expansion. This is precisely the framework that is going to 
be used in this work. 

\

In this chapter, several basic concepts in $\lambda$CDM standard model are going to 
be introduce to finally lead to bayronic acoustic oscillations (BAO). 

%&&&&&&&&&&&&&&&&&&&&&&&&&&&&&&&&&&&&&&&&&&&&&&&&&&&&&&&&&&&&&&&&&&&&&&&&&&&&&&&&&&&&&&&&&&&&&&&&&&&&&&&&&&&&&&&&&&&&&&
\section{ Robertson Walker Metric}
%&&&&&&&&&&&&&&&&&&&&&&&&&&&&&&&&&&&&&&&&&&&&&&&&&&&&&&&&&&&&&&&&&&&&&&&&&&&&&&&&&&&&&&&&&&&&&&&&&&&&&&&&&&&&&&&&&&&&&&

As was mentioned before, observations of the Universe at big scales show 
that it is homogenous and isotropic. This idea is also reinforced by the cosmic 
microwave background radiation (CMB), since it appears to have inhomogeneties
only at very small scales. Nevertheless, it can not be proven and it is taken
as a postulate. 

\begin{itemize}
\item Cosmological principle: \emph{ The Universe is homogeneous and isotropic 
at big scales.}

In this context, homogeneous is understood as the independence of the place 
where a reference system is defined, i.e., the structure of the Universe observed 
is the same no matter the reference system used. 
On the other hand, isotropy establishes that regardless of the direction chosen, 
the same structure is going to be observed. Then, we are dealing with traslational
and rotational symmetry. 

These characteristics are observed on mega parsec scales, i.e., big scales. 
However, this is only valid for the actual epoch, the scale changes with time due 
to the expansion of the Universe. 


\item Weyl postulate :\emph{ Establishes that the geodesics, world lines of 
galaxies, do not intersect except in a singular point in a finite or infinite 
point, past.} 

This one defines a set of observers that move along the geodesics. 
The interception point allows to synchronize watches among different observers,
defining a cosmic time. Therefore, the distance between galaxies can
be measured at the same cosmic time. 

\end{itemize} 

There is another important fact to take into account before speaking of a metric,
the Universe is expanding. It was due to a research on near galaxies performed
by Edwin Hubble, that a redshift was found in most of the galaxies, i.e., they
are moving away from us. 
Considering this movement, one could conclude we are in the center of the 
expansion. But this conclusion is wrong, since the expansion Hubble law is
valid independently where the coordinate system is defined. 

\
A metric that satisfies homogenity and isotropy and additionally
contains a term that corresponds to the expansion is the 
Friedmann Lemaitre Robertson Walker metric. It is defined in 
general terms as $ds^2 = g_{\mu\nu}dx^{\mu}dx^{\nu}$, where $g_{\mu\nu}$
is the metric tensor and uses coordinates $x^{\alpha} = \{ct,x,y,z\}$.
The metric tensor takes the next form $ g_{\mu\nu} = diag\{1,-\frac{a^2}{1-k^2}
-a^2r^2,-a^2r^2\sin^2\theta\}$, and the metric is finally 

\begin{equation}
ds^2= c^2dt^2-a(t)^2\left[\frac{d^2r}{1-Kr^2} +r^2(d^2\theta
 + \sin^2\theta d^2\phi )\right]
\label{metric}
\end{equation} 	

The term $a(t)$ is the scale factor, it describes how the relative
distance between two fundamental observers changes with time. 
The term $K$ is the curvature constant for the actual time and defines
the Universe geometry. When $K=0$ an euclidean metric is recovered 
leading to a flat universe expanding indefinitly. If $K=1$ the Universe
would be described by a spherical geometry and it would collapse because
of its energy matter content. And finally, $K=-1$ corresponds to a
hyperbolic geometry where the Universe would be in accelerated expansion.  

One important aspect to consider is that the geometry depends on the 
total energy matter content, $\Omega_o$. This can be concluded from
te definition of the curvature constant $K = H_o^2(\Omega_o -1)/c^2$. 

Different cosmologies are shown in the figure \ref{factor}. 

%**********************************************************************************************************************
\begin{figure}[htbp]
       \centering
               \includegraphics[width=0.8\textwidth]{Images/chapter2/factordeescala.pdf}
       \caption{ \small Scale factor as a time function. The Universe expansion for 
       different density contributions. A closed Universe is obtained when 
       $\Omega_m = \Omega_o>1$. Also, the WMAP7 parameters show an acelerated expansion. 
        }
       \label{factor}
 \end{figure}
%**********************************************************************************************************************

%&&&&&&&&&&&&&&&&&&&&&&&&&&&&&&&&&&&&&&&&&&&&&&&&&&&&&&&&&&&&&&&&&&&&&&&&&&&&&&&&&&&&&&&&&&&&&&&&&&&&&&&&&&&&&&&&&&&&&&
\section{Hilbert Einstein field equation}
%&&&&&&&&&&&&&&&&&&&&&&&&&&&&&&&&&&&&&&&&&&&&&&&&&&&&&&&&&&&&&&&&&&&&&&&&&&&&&&&&&&&&&&&&&&&&&&&&&&&&&&&&&&&&&&&&&&&&&&

At big scales, the most important fundamental interaction is the 
gravitational one. Hence, the theory of general relativity (TGR) is 
an essential tool in the study of the cosmos. 

At smaller scales, the Newtonian gravitational theory is valid, where, 
the Poisson equation offers a relation between the second derivative
of the field and the source of the field 

\[ \nabla^2\Phi=4\pi G\rho\]

this equation is obtained from TGR for low velocities and 
a weak gravitational field ($\Phi/c^2<< 1$). A key equation of TGR is
the Hilbert-Einstein field equation

\eq{
R_{\mu\nu}-\f{1}{2}g_{\mu\nu}R-g_{\mu\nu}\Lambda = \f{8\pi G}{c^4}T_{\mu\nu}
}

a 6 independent component tensorial equation. The first term of the left
is Ricci tensor (second derivatives of the metric tensor). The second one 
contains the scalar curvature that defines geometry. 
In the third term, $\Lambda$ is the cosmological constant, associated with
the vacuum density term and the accelerated expansion of the Universe.  

In the right side of the equation, the tensor energy-momentum is present.
It includes, as its name suggest, all the contributions to energy and momemtum. 

Hence, the left side of the equation has associated geometry terms, 
while the right one, the ones associated with the matter and energy distribution. 
Then, it could be assure geometry is determined by the matter-energy content 
of the Universe, though, strictly speaking, the energy-momentum tensor depends
in the metric tensor too. 

There is an interesting case of this tensor, when we are dealing with a perfect 
fluid, i.e., without viscosity, homogeneous and isotropic, such it can be expressed
as 

\[T^{\mu}_{\sigma}= diag\{c^2\rho,-P,-P,-P\}\]

where $\rho$ is the density and P is the fluid pressure. 
This shows that not only density causes curvature of space-time 
but also pressure. The Universe can be modelled with this particular
shape of the energy-momentum tensor. 

There are several solutions to the Einstein field equation but not
many in an analytical form, for example, Schwarzchild found the metric
of an estatic spherical mass. Other possible solution is the Kerr 
metric that corresponds to a rotating uncharged mass. The Robertson
Walker metric obviously satisfies these equations. 
	

%&&&&&&&&&&&&&&&&&&&&&&&&&&&&&&&&&&&&&&&&&&&&&&&&&&&&&&&&&&&&&&&&&&&&&&&&&&&&&&&&&&&&&&&&&&&&&&&&&&&&&&&&&&&&&&&&&&&&&&
\section{ Friedmann equations }
%&&&&&&&&&&&&&&&&&&&&&&&&&&&&&&&&&&&&&&&&&&&&&&&&&&&&&&&&&&&&&&&&&&&&&&&&&&&&&&&&&&&&&&&&&&&&&&&&&&&&&&&&&&&&&&&&&&&&&&


From HE field equations and the RW metric is posible to propose 
cosmological models that give account for the observed dynamics
in the Universe. In this direction, the components of the 
field equation can be taken, $\beta=\nu= 0$, time-time component,
and $ii=1,2,3$ (space-time components), from where

\begin{equation}
\frac{\ddot{a}}{a} = -\frac{4\pi G}{3 }\left( \rho + 3 \f{P}{c^2} \right)+\frac{\Lambda c^2}{3} 
\label{fried1}
\end{equation}

\[
\f{\ddot{a}}{a}+2\f{\dot{a}^2}{a^2}+2\f{c^2 K}{a^2} = 4\pi G\left( \rho-\frac{P}{c^2}\right)+\Lambda c^2
\]

\

here, it has been used the energy momentum tensor for an ideal fluid. 
The former expressions are the Friedmann equations and give account 
of Universe expansion dynamics. The terms involved are 
the scale factor $a(t)$ and it is equal to one for the actual epoch,
$a(t_o)=1$, also $\rho$ is the radiation and matter density, $P$ is
the total pressure.  

The equation \ref{fried1} has the form of force equation and it can
be parcially deduced from newtonian mechanic (without the presion
and cosmological constant terms). A most convenient and used form
is obtained after algebraically manipulating them 


\begin{equation}
H(t)=\frac{\dot{a}^2}{a^2}=\frac{8 \pi G}{3}\left(\rho+\frac{\Lambda c^2}{8\pi G}\right) -\frac{Kc^2}{a^2}
\label{fried1}
\end{equation}

this one can be interpretated as an energy equation, where the first term in 
the right hand side is the potential energy. 
This equation also allows to define the Hubble parameter and for the 
actual epoch this coincides with the Hubble constant
$H(t_o)=H_o = 100h$ Km $s^{-1}$ $Mpc^{-1}$.

Aditionally \ref{fried2} can be expressed in terms of the critical
density, i.e.m the matter and energy amount neccesaries for the 
Universe to be flat. Therefore, if the Universe has a bigger density
it would collapse about itself.  Conversely, the Universe would
continue to expanding indefinitly. This quantity is defined as
$\rho_{crit}(t)= 3H(t)^2/8\pi G$.

Dividing \ref{fried2} by the Hubble constant $H_o$ and defining 
the density parameter  $\Omega_{i,o} = \rho_{i,o}/\rho_{crit}(t_o)$ 
with $i=m,r,\Lambda$ is obtained

\begin{equation}
\frac{H^2(z)}{H_o^{2}}=\Omega_{m,o}\left(1+z\right)^3+
\Omega_{r,o}\left(1+z\right)^4+ \Omega_{\Lambda,o} + ( 1-\Omega_o)
\left(1+z\right)
\label{fried2}
\end{equation}


where $\Omega_o=\Omega_{m,o} +\Omega_{r,o}+\Omega_{\Lambda,o}$. 
It has been introduced the relation between redshift and scale
factor $1+z=1/a$. The different contributions to the density 
to the Hubble parameters are observed, i.e., the matter, radiation
and vacuum density. Every component is a function of the 
Universe expansion, although the vacuum energy does not depend
on the redshift, this is, is constant through time. 

\

Initially the Universe was dominated by the radiation, during 
this epoch matter and radiation were coupled, i.e., the De Broglie 
electrons wavelenght were comparable to the radiation one. Because
of this, the photons free mean path is negligible causing the Universe
to be opaque. 

During this coupling, the radiation temperature is equal to the 
matter one and its behaviour is explained as a black body. 

As can be seen in the plot \ref{densidad}, from $z=3230$
matter becomes the major contribution the Universe density.
When $z=1100$ the temperature drop is big enough for the
recombination rate gets higher than the ionization one. 
Recombination refers to the formation of neutral atoms,
that was the ultimate cause to decoupling. 

The last radiation dispertion due to matter still can be
observed, and it is called cosmic radiation background (CMB).
Because of the Universe expansion, its temperature
has been droping, and it is nowadays around $T$ = $2.7K$. 
  
%**********************************************************************************************************************
\begin{figure}[htbp]
       \centering
               \includegraphics[width=0.8\textwidth]{Images/chapter2/density.pdf}
       \caption{ \small Dependence in redshift for $\Omega_\Lambda$, $\Omega_m$ and
       $\Omega_r$. The decoupling between matter and radiation is obtained when 
       $z_{rec}$.
        }
       \label{densidad}
 \end{figure}
%**********************************************************************************************************************

Nowadays, the dominant density component is vacuum, though it is a constant
since it does not depend on the scale factor $\rho_{\Lambda}=-c^4\Lambda/8\pi G$,
in contrast with matter, which depends on it as $a^{-3}$ and radiation as $a^{-4}$,
causing both components diminish in time. 

The cosmological constant is associated to a repulsive force, a opposed behaviour
to gravity, that could give into account the accelerated universe expansion. 

Therea are several solutions to \ref{fried2}, for instance in the Einstein de 
Sitter Universe, there are no radiation or vacuum contributions to the density
and the total density is $\Omega_o=1.0$ In this particular case, the solution
is

\[
t = \f{2}{3H_o}(1+z)^{-3/2}
\]

Therefore, depending on the chosen density values, the equation \ref{fried2}
has different solutions and one can expect several Universe models, i.e.,
depending on the parameters chost, the Universe evolution changes.  
In the case of WMAP9, the parameters used are shown in table\footnote{Table taken from \url{https://lambda.gsfc.nasa.gov/product/map/dr5/params/lcdm_wmap9.cfm}} \ref{WMAPtable}. 

\begin{table}
\begin{center}
  \begin{tabular}{ | c | c | c |}
    \hline \hline
    Parameter & Symbol & Best fit \\ \hline \hline 
    Hubble constant $(km/Mpc-s)$ & $H_0$ & 68.65$\pm$ 0.93 \\ \hline
    Baryon density & $\Omega_b h^2$ & 0.02248$\pm$0.00044 \\ \hline
    Cold dark matter density & $\Omega_c h^2$ & 0.1165$\pm$0.0024 \\  \hline
    Dark energy density & $\Omega_\Lambda$ & 0.705$\pm$0.0011\\ \hline
    Scalar spectral index & $n_s$ & 0.967$\pm$ 0.01 \\ \hline
    Sigma 8 & $\sigma_8$& 0.830 \\ \hline
  \end{tabular}
    \caption{ Fit cosmological parameters from WMAP+BAO nine-year results.}
  \label{WMAPtable}
\end{center}
\end{table}

Other possible Universe models are for example, one obtained when matter
density is the only contribution, and different total universe density.
If this former is bigger than one, the Universe obtained is closed. 
Other, it is one obtained when the universe is dominated for the vacuum
contribution. In this case, the Universe is always open. When all the 
contributions are present, the Universe can be open or closed depending on
total density paremeter.

%&&&&&&&&&&&&&&&&&&&&&&&&&&&&&&&&&&&&&&&&&&&&&&&&&&&&&&&&&&&&&&&&&&&&&&&&&&&&&&&&&&&&&&&&&&&&&&&&&&&&&&&&&&&&&&&&&&&&&&
\section{ State equation }
%&&&&&&&&&&&&&&&&&&&&&&&&&&&&&&&&&&&&&&&&&&&&&&&&&&&&&&&&&&&&&&&&&&&&&&&&&&&&&&&&&&&&&&&&&&&&&&&&&&&&&&&&&&&&&&&&&&&&&&

As mentioned before, scale factor determines the Universe expansion, 
hence it is mandatory to find relations that relates the different universe 
density components with it. 

Assuming matter is an isolated system, the first law of thermodynamics is
expressed as $dU = -pdV$, where relativistic terms are included in the 
internal energy term. Using the equipartition theorem and derivativing 
internal energy with respect to scale factor is obtained 

\[ T \propto a^{-2} \]

but from the equation state $P = NkT$ and taking into account that $N = N_oa^{-3}$  
it is known that  $P \propto a^{-5}$. Pressure due to matter diminishes 
strongly with universe expansion, while density and temperature change smoother. 
The latter is another cause for vacuum to dominate universe expansion. 

\

Radiation energy density is

\[\xi=\sum_{\nu}N(\nu)h\nu\]

where $N(\nu)$ is photon density and satifies the relation $N \propto (1+z)^3$,
so that $\xi \propto  \sum_\nu C_\nu a^{-4} $. Comparing with Stefan Boltzmann
law is concluded that $T\propto a^{-1}$. Radiation pressure dependence on scale
factor is found using $ P=\frac{1}{3}\epsilon_{total}$ with which is obtanied
$P \propto a^{-4}$. 

Otherwise, vacuum satisfies $\epsilon_{total}=\rho c^2$ where $\rho$
is an effective density. Replacing this result in first law of thermodynamics
and derivativing with respect to scale factor  

\[ P=-\rho c^2 = -\f{ \Lambda c^4}{8\pi G}\]

the vacuum density constancy has to be used in its deduction.

%&&&&&&&&&&&&&&&&&&&&&&&&&&&&&&&&&&&&&&&&&&&&&&&&&&&&&&&&&&&&&&&&&&&&&&&&&&&&&&&&&&&&&&&&&&&&&&&&&&&&&&&&&&&&&&&&&&&&&&
\section{ Perturbation evolution in the newtonian regimen }
%&&&&&&&&&&&&&&&&&&&&&&&&&&&&&&&&&&&&&&&&&&&&&&&&&&&&&&&&&&&&&&&&&&&&&&&&&&&&&&&&&&&&&&&&&&&&&&&&&&&&&&&&&&&&&&&&&&&&&&

As already stated, there is no radiation comming toward us from a epoch 
previous to decoupling. Although, due to last scattering between radiation
and matter, highly homogeneous and isotropic distribution of matter is
observed, i.e., patterns obtained from background cosmic radiation\footnote{
Image WMAP taken from \url{http://lambda.gsfc.nasa.gov/product/map/current/m_images.cfm}} (Figure \ref{CMB}).  
The wavelength associated to this cosmic radiation is in the microwave range.
In CMB radiation, small temperature perturbations are observed, hence, small matter
perturbations were present at this epoch. These are the initial seeds from where 
structures observed nowadays formed. 

In this structure growth, density fluctuations are increasing but it is not
until they got a size of $\delta\sim 1$ that their movement was not exclusive
due to cosmic expansion. The fluctuations have grown enough to start talking 
about galaxy formation when their density gets around $1\times 10^6$ compared with
the background density, this happens for a epoch around $z\sim 100$.

But, it is still important to study the initial stages of the fluctuations.
Because of this, a linear regime treatment for fluctuations when $\delta\ll 1$
are key in such a study. 

%**********************************************************************************************************************
\begin{figure}[htbp]
       \centering
               \includegraphics[width=0.6\textwidth]{Images/chapter2/CMB.png}
       \caption{\small Cosmic background radiation image obtained by WMAP 
       using 5 different maps.}
       \label{CMB}
 \end{figure}
%**********************************************************************************************************************


%######################################################################################################################
\subsection{ Newtonian description  }
%######################################################################################################################


Inhomogeneities have to be formed at initial stages, these initial density fluctuations 
have a characteristic length much smaller than the hubble radious. This implies that the
size of the fluctuations is very small compared with scales where the universe curvature 
is significant, making this newtonian approximation be valid. But because of this, casualty 
is taken as instantaneous. 

Taking into account the last shown, a fluid in motion of density $\rho$ subject to 
a gravitational field that suffers changes in pressure satifies 

\begin{eqnarray}
\der{\rho}{t}&=& -\rho \nabla_r\cdot \textbf{u} \nonumber\\
\der{\textbf{u}}{t} &=& -\f{\nabla_r P}{\rho}-\nabla_r\phi \nonumber\\
\nabla^2_r \phi & =& 4\pi G\rho 
\label{euler}
\end{eqnarray}

As density fluctuations are more of our interest, since inhomogeneties 
respect to background are the ones that trigger potential wells, it is
useful to expresss density as $\rho = \overline{\rho}+\delta\rho$, where
$\overline{\rho}$ is the background density. 
Here, it is necessary to clear another point. Particle velocities have 
two different contributions, the first one is caused because of the Universe
expansion and the other one is the proper velocity of the particle,recessional 
and peculiar velocities respectively. 
From the latter mentioned, the coordinate system can be changed from \ref{euler}
an euler description to a lagrangian one, i.e., moving with the Universe expansion.
Let's see this in more detail, velocity in a eulearian description is 
$\textbf{u}= a\dot{\textbf{x}}+ \textbf{x}\dot{a} = \textbf{v}+\textbf{x}\dot{a}$, 
where $\textbf{v}$ is the peculiar velocity and $\textbf{x}\dot{a}$ is the universe 
expansion velocity. 
Then, making a change to comovil coordinates, coordinates that move with the universe
expansion and changing density to density contrast, the next equations are found

\begin{eqnarray}
\pder{\delta}{t} &=& -\f{1}{a}\nabla\cdot[(1+\delta)\textbf{v}]\nonumber\\
\pder{\textbf{v}}{t}+\f{\dot{a}}{a}\textbf{v}+\frac{1}{a}(\textbf{v}\cdot\nabla)\textbf{v}& = &
-\f{\nabla \Phi}{a}-\f{\nabla P}{a\overline{\rho}(1+\delta)}\nonumber\\
\nabla^2 \Phi &=& 4\pi G\overline{\rho}a^2\delta
\label{ec.movimiento}
\end{eqnarray}

the first one corresponds to the continuity equation, the second one is Euler's 
equation and the last one is poissonian gravitational field equation. Velocity 
components appear due to gravitational interactions and changes in pressuare,
here $\Phi$ is an effective potential. 

\

Aditionally, equation of state relating the thermodynamic quantities P, 
$\rho$ and s( entropy ) for this cosmological fluid is 

\begin{equation}
P(\rho,s)= \left[\f{h^2}{2\pi(\mu m_p)^{5/3}}e^{-5/3}\right]\rho^{5/3}\exp \left(\frac{2}{3}\f{\mu m_ps}{k_B}\right)
\label{ec.estado}
\end{equation}

Manipulating algebraically the continuity equation, Poisson equation and state equation,
a wave equation for density fluctuations can be obtanied

\begin{equation}
\f{\partial^2\delta}{\partial t^2}+2\f{\dot{a}}{a}\pder{\delta}{t} =
4\pi G \overline{\rho}\delta + \f{C_s^2}{a^2}\nabla^2\delta+\f{2}{3}\f{\overline{T}}{a^2}\nabla^2s
\label{onda}
\end{equation}

where $\overline{T}$ is the background temperature and $C_s$ is sound speed.
The Universe expansion is seen in the second term in the left side. Since for an expanding
universe the term $\dot{a}{a}$ is positive, Hubble param, its effect is opposed to the perturbation 
growth. This result was expected due to expansion is against collapse leaving to
a decrease in growth. 	

In the right side causes for perturbation evolution are shown, these can make them grow 
or disipate. Entropy can be considered as heat interchange between perturbation and 
surroundings, causing the expansion or growth of the perturbation. Gravitational field
is a source for perturbation growth. 

A solution to the perturbation equation in terms of Fourier series is proposed

\begin{eqnarray}
\delta(x,t) &=& \sum_k \delta_k(t)e^{ik\cdot x} \nonumber\\
s(x,t) &=& \sum_k s(t)e^{ik\cdot x} \nonumber
\end{eqnarray}

an important aspect is the indepencency of the functions $e^{ik\cdot x}$ allowing 
equation \ref{onda} be expressed as 

\begin{equation}
\f{d^2\delta_k(t)}{dt^2}+2\f{\dot{a}}{a}\der{\delta_k(t)}{t} = 
\left[4\pi G\overline{\rho}-\f{C_s^2 k^2}{a^2}\right]\delta_k(t)
-\f{2}{3}\f{\overline{T}}{a^2}k^2s_k(t)
\label{ec.modos}
\end{equation}

the equation solution provides expansion coefficients for Fourier series,
from where, the behaviour to density fluctuations, their growth or 
disipation, is obtained. 

%######################################################################################################################
\subsection{ Jeans Inestability}
%######################################################################################################################

Before solving the mode equation \ref{ec.modos}, it is important to develop
some intuition about physical phenomena. This can be achived making some 
simplifications. For example, taking an isentropic static universe ($\dot{a}=0$)
the expression becomes

\[
\f{d^2\delta_k(t)}{dt^2} +
\omega^2\delta_k(t) = 0
\]

with $\omega^2 = C_s^2k^2/a^2-4\pi G\overline{\rho}$. Clearly the solution of mode 
equation depends on $\omega$'s sign, if $C_s^2k^2/a^2>4\pi G\overline{\rho}$,  
$\omega$ is positive and the solution obtained is oscillatory. In other words, 
this solution is a sound wave not gravitationally inestable, therefore, it is
not of our interest. 
By the other side, if $4\pi G\overline{\rho}>C_s^2k^2/a^2$ the solution takes
the form $\delta_k(t)\propto e^{\Gamma_k t}$, with $\Gamma_k=i\omega_k$ named
growth rate. In this case, the perturbation disipates or collapses depending
on the square frequency sign chose. 

Physically, it is expected fluctuations tend to collapse 
because of gravity though preassure gradient caused by atomic
interactions goes against it. The ones of interest are those
ones that collapse. For this, a minimum length that a perturbation
must have to obtain an inestable flutuaction, i.e., a collapsing
perturbation, is defined. Hence, from the frequency Jeans' length
is obtained $\lambda_J = 2\pi a/k_j$ and satisfies 
$\lambda_J = 2\pi/k_j = C_s(\pi/G\rho)^{1/2}$.
The growth rate can be rewritten in terms of $\lambda_J$, 
allowing to make the next comparison, if $\lambda_{pert}>>\lambda_J$
is satisfied the perturbation collapses where $\lambda_{pert}$
is the length of the perturbation.


As was shown previously, Jeans' length depends on sound speed 
that is define as 

\[
C_s^2 = \left( \pder{P}{\rho} \right)_s
\]

and using also the state equation \ref{ec.estado} a relation 
is found. It was also used the fact that radiation and matter 
temperature are equal for $z \leq z_{eqv}$ due to coupling. 
But after decoupling every component evolves independently. 
Jean's and mass length, the last one defined as $M_J =  \pi\overline{\rho}_{m,o}\lambda_J^3 $,
are given by

\begin{eqnarray}
\lambda_J \approx 0.01(\Omega_{b,o}h^2)^{-1/2}Mpc\\
M_J \approx 1.5\times 10^5 (\Omega_{b,o}h^2)^{-1/2}M_{\odot}
\end{eqnarray}

Before decoupling, speed of sound was affected not only 
by matter but for radiation, even the latter one was more
important since radiation density dominates in this epoch
(figure \ref{densidad}). 
The valid state equation for radiation in this epoch is
$P = c^2\rho_r/3$ and it can be shown that the change 
in magnitude order evaluated before and after decoupling
of Jean's length and mass is $2.6\times 10^{-5}$ and $1.8\times 10^{-14}$
respectively. From previous asseveration a possible conclusion is
decoupling increases gravitational collapse since  minimum characteristic length
required for collapsing becomes smaller. 

\

Until now dark matter has not been mentioned, a component that contrary
to baryonic matter does not interact with radiation. But it is around
$23\%$ of the overall Universe content, making it responsible for the 
large scale mass distribution observed. Dark matter particles interact
among them and with baryonic matter through gravitational interaction.
This leads to dark matter potential wells where baryonic matter can also fall.  
It occurs around $z \sim 3500$. But, dark matter initial seeds for potential 
wells are formed before decoupling, since they do not interact with radiation,
they can gather forming them. While this happens, baryonic matter and radiation 
interacts gravitationally with dark matter, leading to no formation of 
baryonic seeds precisely because of the mentioned coupling. Later a more detailed 
explanation will be provided. 

The last arguments exposed allow to express perturbations as $\delta = \delta_oD(z)$,
where $\delta_o$ is a initial seed from where a dark matter potential well formed
and $D(z)$ is a growing function of a initial seed.
Then, rewritting equation \ref{ec.modos}  

\[
\f{d^2D(z)}{dz^2}+\left[ \f{H'(z)}{H(z)} -\f{1}{1+z}\right]\der{D}{z}
= \f{1}{H^2(z)}\left[ \f{4\pi G\overline{\rho}(z)}{(1+z)^2} \right]D(z)
\]

%**********************************************************************************************************************
\begin{figure}[htbp]
       \centering
               \includegraphics[width=0.8\textwidth]{Images/chapter2/masavacio.pdf}
       \caption{\small Perturbation evolution for a model Mass-vacuum, the behaviour
       for different contributions of each component. }
       \label{masavacio}
 \end{figure}
%**********************************************************************************************************************

in the expresion isotropy was assumed and the velocity term was neglected
since the perturbations of our interest satify $\lambda_J \ll \lambda$.
Using the last equation, different solutions changing density parameters
are obtained. For example, an Einstein de Sitter Universe  $\delta_k(z)=\delta_o(1+z)^{-1}$,  
an Universe dominated by radiation $\delta_k(z)=\delta_o(1+z)^{-1.22}$ and
an Universe dominated by vacuum $\delta_k(z)=\delta_o(1+z)^{-0.58}$. 
Obviously, for different contributions to every component of density can be found,
in figure \ref{masavacio} is shown perturbations evolution for a model with
matter and vacuum contributions. As expected, for bigger matter density
perturbations grow faster while for universes with bigger vacuum contribution, 
perturbations grow slower cause accelerated expansion. In the latter, a bigger
initial mass is required to start collapsing. In the figure, largest redshift
is $z=0$ but solution is only valid while $\delta\ll 1$. 

%&&&&&&&&&&&&&&&&&&&&&&&&&&&&&&&&&&&&&&&&&&&&&&&&&&&&&&&&&&&&&&&&&&&&&&&&&&&&&&&&&&&&&&&&&&&&&&&&&&&&&&&&&&&&&&&&&&&&&&
\section{ Higher order perturbation theory }
%&&&&&&&&&&&&&&&&&&&&&&&&&&&&&&&&&&&&&&&&&&&&&&&&&&&&&&&&&&&&&&&&&&&&&&&&&&&&&&&&&&&&&&&&&&&&&&&&&&&&&&&&&&&&&&&&&&&&&&

As mentioned, perturbations start growing after decoupling. This growth leads
precisely to an increase in their density contrasts. 
When they have increased enough, around $\delta\sim 1$, Newtonian description 
stops being valid. A first approach to study such perturbations is spherical collapse. 
Even for those $\delta$ values, perturbations are not considered an objetc. 
In such cases, gravitational interaction determines dynamics more than Universe
expansion. This ocurrs when density gets around $\sim 100 \overline{\rho}$. 

Now, let's see in more detail spherical collapse. In this scenario a perturbation
has spherical symmetry and pressure exerted among particles is neglectable. 
But the latter supposition would imply that collapsing cause perturbations 
to end up as a singularity, since there is no force against collapse.
If tidal forces are included it would not be the case but it is taken 
as a first approximation.

Time calculated for a perturbation to collapse is 

\[
1+z_c = \f{1+z_{max}}{2^{2/3}}
\]

where $z_{max}$ is redshift when perturbation dynamics is no longer affected by universe
expansion. Collapsing time for a perturbation is smaller than time necessary to become 
a gravitationally linked object. Aditionally redshift estimative for perturbation
satisfies virial theorem is

\[
(1+z_{vir})\leq 0.47\left( \f{v}{100kms^{-1}}\right)^2\left( \f{M}{10^{12}M_{\odot}}\right)^{-2/3}(\Omega_oh^2)^{-1/3}
\]

it can be noticed from this expresssion that more masive objects and with a bigger 
dispersion velocity has a bigger viralization time. 

%######################################################################################################################
\subsection{ Zeldovich approximation }
%######################################################################################################################

The last model for non lineal perturbartion treatment 
has several failures. A next step in this direction is 
Zeldovich approximation that provides a more realistic description, 
since it considers that fluctuations' shape is an ellipsoid.
Though pressureless and without viscosity fluid supposition are still taken.
Coomovil coordinates satisfies 

\[
\textbf{x} = a(t)\textbf{r}+b(t)\textbf{p}(\textbf{r})
\]

where $\textbf{r}$ are eulerian coordinates and $b(t)\textbf{p}(\textbf{r})$
is displacement respect to initial position. The last term is related 
with the gravitational potential generated by initial perturbations. 
It can be shown that initial perturbations are given by 

\[
\rho = \frac{\overline{\rho} }{[1-\alpha b(t)][a(t)-\beta b(t)][a(t)-\gamma b(t)]}a^3(t)
\]

where $\alpha$, $\beta$ and $\gamma$ account for the expansion or contraction 
along the three main axis of the ellipsoid. 
When  $\alpha \geq \beta \geq \gamma$ and $b(t)$ are sufficiently big enough a 
singularity appears. If $\alpha$ also gets its highest value causes that fluctuations
get a pancake shape due to a faster contraction in x axis. 
Despite of the simplicity, it is precise model accurate with numerical results. Hence,
it is used in observational cosmology and initial condition generation in N body
simulations. 

%&&&&&&&&&&&&&&&&&&&&&&&&&&&&&&&&&&&&&&&&&&&&&&&&&&&&&&&&&&&&&&&&&&&&&&&&&&&&&&&&&&&&&&&&&&&&&&&&&&&&&&&&&&&&&&&&&&&&&&
\section{ Statistical properties of cosmological perturbations }
%&&&&&&&&&&&&&&&&&&&&&&&&&&&&&&&&&&&&&&&&&&&&&&&&&&&&&&&&&&&&&&&&&&&&&&&&&&&&&&&&&&&&&&&&&&&&&&&&&&&&&&&&&&&&&&&&&&&&&&

Nevertheless, to study the evolution of the universe since decoupling requiere to know 
or assume a density field, to know density in every point and its evolution with time.  

While in linear regime, there are a huge amount of perturbations described by different 
fourier modes. Since they evolve independently their amplitudes change with time, which 
can be modelled using a transfer function and a linear growth rate. To characterize
such amount of density field values, statistical properties could be defined. This is,
not considering individual positions or properties but instead moments defined from
some distribution function. This idea is supported by the fact that there is no access 
to the primordial perturbations that originated the large scale structure observed 
nowadays. Hence, our Universe could be considered as a realization of a random process 
where a statistical treatment results as a natural way to study it. 

This statistical characterizations leads to the study of a joint probability 
distribution where the moments, in principle infinite, would describe the 
cosmic density field. 
Following this approach, the probability of having a mode between $\delta_k$ and 
$\delta_{k}+d\delta_k$ is, 

\begin{equation}
\mathcal{P}(\delta_{\mbox{\boldmath$\kappa$}})r_{\mbox{\boldmath$\kappa$}}dr_{\mbox{\boldmath$\kappa$}}d\phi_{\mbox{\boldmath$\kappa$}}
=
\exp\left[-\f{r_{\mbox{\boldmath$\kappa$}}^2}{2V_u^{-1}P(\kappa)}\right]	\f{r_{\mbox{\boldmath$\kappa$}}}{V_u^{-1}}
\f{dr_{\mbox{\boldmath$\kappa$}}}{P(\kappa)}\f{d\phi_{\mbox{\boldmath$\kappa$}}}{2\pi}
\label{probabilidad}
\end{equation}


where $r_{\mbox{\boldmath$\kappa$}}$ corresponds to perturbations amplitude,  $\phi_{\mbox{\boldmath$\kappa$}}$ 
is the phase and varies between $[0,2\pi)$.  The joint probability distribution function 
is useful because it allows the independence of the terms $\delta_{\mbox{\boldmath$\kappa$}}$, 
or in other words it is the product of every mode

\[
\mathcal{P}_{\mbox{\boldmath$\kappa$}}(\delta_{\mbox{\boldmath$\kappa$}1},...,\delta_{\mbox{\boldmath$\kappa$}N})=
\prod_{\mbox{\boldmath$\kappa$}}\mathcal{P}_{\mbox{\boldmath$\kappa$}}(\delta_{\mbox{\boldmath$\kappa$}})
\]


This expression is not satisfied when the inverse fourier transform is done, since the 
probability density is not separable in the initial coordinate space. The term $P(\kappa)$ 
(assuming isotropy in \mbox{\boldmath$\kappa$}) is the power spectrum defined in the 
fourier space and it is related to the 2 point correlation function in the real space

\begin{equation}
P(k) = \f{4\pi}{V_u}\int_0^\infty \xi (r)\f{sin(kr)}{kr}r^2dr =  \langle|\delta(\mbox{\boldmath$\kappa$})|^2\rangle
\label{pk}
\end{equation}

as it can be seen in the expression, the isotropy of the universe is taken into account since during the power spectrum calculation, an average is done over all  possible orientations of the vector ${\mbox{\boldmath$\kappa$}}$

Here, $\xi (r)$  describes the distribution of points, telling for example, how clumpy certain region is or in general the clustering properties of the system. Being more specific, this function describes the excess probability of finding a particle at a distance $r$ from a particle selected at random over the expected in an uniform, random distribution. 
The two point correlation is related to the density contrast as

\[\xi (r) =  \langle  \delta(\textbf{x})\delta(\textbf{x}+\textbf{r}) \rangle\]

therefore, the density field leads to a direct way to find the power spectrum through a fourier 
transform. $\xi$ only depends on the amplitude of r because of the assumption of homogeneity
and isotropy. 

So far, it has been shown two statistical measures, a fourier pair, in real space the 
correlation function and in fourier space the power spectrum. 
But other moments can be specified as previously mentioned, in general an l point 
correlation function can be defined through the next expression 
$\xi^l(\vec{x}_1,\vec{x}_2,\ldots \vec{x}_l) \equiv  \langle  \delta_1\delta_2 \ldots\delta_l)\rangle $ where the connected terms are the ones that contributed to the calculation.

An important remark is that if initial density pertubations follow a gaussian distribution 
all moments higher than two (2 point correlation function) are zero, i.e., the density
perturbation field is completely described by the two first moments of the distribution. 

\


%**********************************************************************************************************************
\begin{figure}[htbp]
       \centering
               \includegraphics[width=0.95\textwidth]{./Images/chapter2/PS_CF.png}
       \caption{\small BAO peak in the correlation function in the left and the oscillations of BAO in the power spectrum in the right,
       (\cite{PLOT}).  
       The lower curve is the main SDSS sample and the upper one is the LGR sample. }
       \label{ps_cf}
 \end{figure}
%**********************************************************************************************************************

It is usually considered for the initial density field, that density contrasts 
follow a normal distribution centered in $\langle \delta \rangle = 0$. This idea is supported by
inflationary scenarios where a random gaussian perturbation field arises naturally from quantum 
fluctuations during inflation. Since there are a large number of modes, the central limit theory 
would support this idea if the mode phases are independent. One of the advantages of this model 
is that the perturbation field remains gaussian during linear evolution. 

Additionally it has been found that the initial power spectrum expected from inflation theories 
has the form $P(\kappa)= k^n$. If $n=1$ the power spectrum is called Harrison-Zeldovich which is 
commonly used. 

With the initial power spectrum it can be done an inverse fourier transform and this way creating the initial density field. 

Using inflationary models the shape of the power spectrum is well determined but there are not
amplitude predictions, i.e., there is not a defined normalization of the power spectrum. 
One commonly way to do such thing is through the variance of the galaxy distribution when
sampled with randomly placed spheres at radius R. The relation between the variance of the 
density field and the power spectrum is

\[
\sigma^2(R) = \f{1}{2\pi^2} \int P(k)\hat{\omega}_R(k)^2k^2 dk
\]

where $\hat{\omega}_R(k)$ is the Fourier transform of the spherical top hot model.

\[\hat{\omega}_R(k) = \frac{3}{(k R)^2}[\sin(kR)-kR\cos(k R)]\]
 
In this approach $\sigma(R)$ is taken as one when $R=8h^{-1}Mpc$ because of 
the galaxy distribution. 
Then, normalizing the power spectrum would imply to force $\sigma(R)$ to be one 
for the mentioned distance.

But several problems arise, one is that this normalization is not precisely valid
for linear regime since $\sigma(R)\approx 1$ when $\delta(R)\ll 1$. Other one is
baryonic matter is probably a bias tracer of the mass distribution. 



\

It is necessary to consider that after recombination epoch, density contrasts started
growing in size causing a non linear growth to appear. This implies a change in the 
density field and likewise the power spectrum. 



%######################################################################################################################
\subsection{ Clustering in the real and redshift space }
%######################################################################################################################


%-----------------------------------------------------------------------------------------------------------
\subsubsection{ Redshift distortions }
%-----------------------------------------------------------------------------------------------------------
%-----------------------------------------------------------------------------------------------------------
\subsubsection{ Real space correlation functions }
%-----------------------------------------------------------------------------------------------------------

%&&&&&&&&&&&&&&&&&&&&&&&&&&&&&&&&&&&&&&&&&&&&&&&&&&&&&&&&&&&&&&&&&&&&&&&&&&&&&&&&&&&&&&&&&&&&&&&&&&&&&&&&&&&&&&&&&&&&&&
\section{ Baryonic acoustic oscillations }
%&&&&&&&&&&&&&&&&&&&&&&&&&&&&&&&&&&&&&&&&&&&&&&&&&&&&&&&&&&&&&&&&&&&&&&&&&&&&&&&&&&&&&&&&&&&&&&&&&&&&&&&&&&&&&&&&&&&&&&
Let us consider the epoch before recombination. The baryonic plasma (ionized protons and electrons) was coupled with radiation via 
Thomson scattering,  i.e., the electric field of photons accelerate charged particles making small density contrast to disperse. Nevertheless,
considering that dark matter do not interact with radiation, small dark matter density contrasts can form. Hence, the baryons are 
subject to two competing forces, radiation pressure and gravitation. Consider a particular dark matter density contrast that attract
nearby baryons, they start clustering around the dark matter forming a bigger density contrast. But, due to the pressure caused
by coupling, the outward force becomes bigger than gravity, making baryons to move outward as a sound wave. This oscillation 
of the baryonic plasma is known as baryonic acoustic oscillation. 

\

When decoupling occurs and temperature drops, the force responsible for the expansion of the shell disappear, this is, 
the pressure caused by the coupling between baryons and radiation, leading baryons in the last position they were located. 
The scale of the baryonic acoustic oscillation is 
usually called the sound horizon and it can be computed as

\[  
s = \int_{z_{rec}}^{\infty} \f{c_s dz}{H(z)}
\]


where $c_s$ is the velocity of the propagation and $H(z)$ is the hubble param, (\cite{pilar}). 

Therefore, there is a spherical shell formed 
around the dark matter density contrast. Now, not only dark matter density contrast seed gravitational instability but 
the baryons on the shell. 

The structures continue to grow reaching non linear growth and wipe out the imprint lead by the baryonic acoustic
oscillations except for the bigger ones. The estimated size of the remaining BAO is $105$Mpc, causing that scale to 
be more likely to have galaxy formation activity. The reason this distribution is not observed at cosmological scales is because
of the amount of imprints, they smear out the preferred scale. Though, it is expected an enhancement in the two point
correlation at scales of the baryonic oscillations. As was seen before, there is a direct correspondence between the 
two point correlation and the power spectrum. Hence, a characteristic oscillation in the power spectrum caused by the imprint of BAO is naturally found. 	

 
%**********************************************************************************************************************
\begin{figure}[htbp]
       \centering
               \includegraphics[width=1.0\textwidth]{Images/chapter2/BAOS2.png}
       \caption{\small All the figures show the evolution of the radial mass profile of dark matter,  baryons, photons
       and neutrinos. First one: Initial perturbations of the four species. 
       Second one: the neutrinos do not interact and move away, 
       the plasma of baryons and radiation overdensity expands because of radiation pressure, the dark matter
       continues to fall in the perturbation. Third one: The temperature drops enought to lead to decoupling, 
      the baryons slows down until stopped, the radiation and neutrinos continue moving away. Fourth one:  
      The dark matter and baryons eventually get the same distribution because of the gravitational interaction.}
       \label{DBM}
 \end{figure}
%**********************************************************************************************************************

Since baryonic acoustic oscillations are primarily a linear phenomenon, they are preserved in the power spectrum despite of 
the temporal evolution. Then,  BAO are used as an standard ruler, specifically for high redshift where other rulers tend to fail. 
This is commonly used for constraining dark matter models. 

\

Although, the nonlinear collapse change the shape and position of baryonic acoustic oscillations, broaden and shift the 
peak. This is clearly seen in the figure (\ref{peak}), the broadening of the peak initially shown as a very sharp causes 
a damping in the frequency in the power spectrum. It is expected that this effect that is going to be studied in this work, affects baryonic 
acoustic oscillations on scales around $\sim 10$Mpc.

The diffusion damping (silk damping)  also causes a reduction in size of density inequalities by the diffusion of photons from 
hot regions to colder ones during the epoch of recombination. 

%**********************************************************************************************************************
\begin{figure}[htbp]
       \centering
               \includegraphics[width=0.9\textwidth]{Images/chapter2/width.png}
       \caption{\small In the left the correlation function and the right the power spectrum. When the width of the peak is increased
       the acoustic oscillation obtained in the power spectrum is damped.    }
       \label{peak}
 \end{figure}
%**********************************************************************************************************************
