\chapter{ Cosmological simulations }


In cosmology one issue that currently exists is finding 
the dynamical evolution of the Universe, from its first 
stages to the actual epoch. This makes necessary tools to
study a big amount of interacting particles. 
Considering all possible components in the Universe evolution, 
the particles required to simulate the interaction among them
would be so huge that it would not be viable
to predict the evolution of every single particle for every
time step. 

In cosmological simulations a key component to consider
is dark matter, since it is mostly because of this one that 
the Universe has a filament structure. 
The latter asseveration is due to dark matter dominates the gravitational
interaction, not only because of its amount compared to
baryonic matter but also because it only interacts in this way.

This component cannot be observed directly cause it does not interact 
with radiation. But, since its gravitational effects are strong, there are 
observational evidence that accounts for its existence. For example, 
observing in galaxy rotation curves, the velocity measured at the outskirts 
was too fast to explain only with baryonic matter. 

A key argument to ignore baryonic matter in simulations is that
defining as total density the sum of baryonic and dark matter only, 
dark matter would contribute with around $80\%$ of all the density
content. 

The next chapter is divided in several sections, the first one
corresponds to the methods used in cosmological simulations to
calculate the gravitational evolution of the system. The second
one contains different criteria selection to detect a dark matter
halo in simulations. The third and fourth sections are dedicated
to explain how to build two different statistical measures of clustering
in real and fourier space, the correlation function and power spectrum 
respectively. 


%&&&&&&&&&&&&&&&&&&&&&&&&&&&&&&&&&&&&&&&&&&&&&&&&&&&&&&&&&&&&&&&&&&&&&&&&&&&&&&&&&&&&&&&&&&&&&&&&&&&&&&&&&&&&&&&&&&&&&&
\section{ Numerical methods }
%&&&&&&&&&&&&&&&&&&&&&&&&&&&&&&&&&&&&&&&&&&&&&&&&&&&&&&&&&&&&&&&&&&&&&&&&&&&&&&&&&&&&&&&&&&&&&&&&&&&&&&&&&&&&&&&&&&&&&&


In the last chapter the equations that describe the evolution of 
the perturbations in linear regime were provided and the Zeldovich 
approximation for non linear regime was also given. But in a cosmological
simulation is necessary initial conditions for the density field
or a initial shape of the power spectrum. In these cosmological simulations 
dark matter is a key component, in many cases is the only particle considered,
hence some observational evidences are going to be provided. As already stated
dark matter does not interact with radiation, the reason to be proposed 
is the gravitational effects found in different systems studied. It is 
about the  $\sim 24\%$ content of the Universe therefore it is important 
to study its evolution. 

One of the first observational evidence was found in the Coma cluster due to
a mass estimation from the virial theorem, let's see this in more detail  

the specific kinetic energy of the system is $T = v^2/2 \sim 3\sigma^2/2 $ 
where $\sigma$ is the galaxy velocity dispertion and the potential energy
is $U = 3GM_{vir}/(5R_{vir})$. From the mass-luminosity ratio and the 
mean luminosity of the cluster, a second estimative of the mass is found. There is
a discrepancy between the two values so big to assert that $\sim 90\%$ of the 
cluster's mass is not visible. 

The rotation curve of the galaxies can be other prove for dark matter existence,
for example, the velocity measures performed with respect to the radious of Andromeda 
galaxy (or another spiral galaxies) is approximately the same independent of the radial
distance of the stars to the center of the galaxy. From this it could be affirmed that
density is uniform along the galaxy contrary to expected for the observed 
number of star in function of the radious. 


The Bullet cluster is composed by two two clusters that are colliding, an event 
not commonly observed. The gas of them reaches velocities around $\sim 10$ $millions$ $of$ $miles/h$
during the violent collition while they interact among them because of their charge. 
This interaction disminishes the gas velocity but this does not happen with dark
matter cause it does not interact electrically. 
Using gravitational lensing a distortion map is obtained. Using the X rays detected
and the distortion map four different groups of matter are found, 2 bigger ones
that correspond to the dark matter component and two smaller ones that correspond
to luminous matter formed from the intercluster gas. These presents a strong
evidence of dark matter existence. 

\

Additionally to the initial conditions, the box size $L$ and the number of 
dark matter particles $N^3$ that would be used should be fixed. 
The gravitational interaction calculation of such a big number of particles
could in principle be calculated through direct calculation. This first
attempt is not very efficient or even it is not possible to perform it
since the computing time or the computational resources would be very 
big to be viable. The latter is the reason for approximate methods 
to appear as a possible solution that implies more reasonable computing times.

A main objective in a simulation could be the study the formation process, fusion and
further interactions that are produced among halos and vacuum regions that conform
the filamentary structure of the Universe. 


%&&&&&&&&&&&&&&&&&&&&&&&&&&&&&&&&&&&&&&&&&&&&&&&&&&&&&&&&&&&&&&&&&&&&&&&&&&&&&&&&&&&&&&&&&&&&&&&&&&&&&&&&&&&&&&&&&&&&&&
\section{ Halo selection }
%&&&&&&&&&&&&&&&&&&&&&&&&&&&&&&&&&&&&&&&&&&&&&&&&&&&&&&&&&&&&&&&&&&&&&&&&&&&&&&&&&&&&&&&&&&&&&&&&&&&&&&&&&&&&&&&&&&&&&&
%######################################################################################################################
\subsection{ Friends of friends }
%######################################################################################################################
%######################################################################################################################
\subsection{ Bound density maximum }
%######################################################################################################################



%&&&&&&&&&&&&&&&&&&&&&&&&&&&&&&&&&&&&&&&&&&&&&&&&&&&&&&&&&&&&&&&&&&&&&&&&&&&&&&&&&&&&&&&&&&&&&&&&&&&&&&&&&&&&&&&&&&&&&&
\section{ Correlation functions in cosmological simulations }
%&&&&&&&&&&&&&&&&&&&&&&&&&&&&&&&&&&&&&&&&&&&&&&&&&&&&&&&&&&&&&&&&&&&&&&&&&&&&&&&&&&&&&&&&&&&&&&&&&&&&&&&&&&&&&&&&&&&&&&



%&&&&&&&&&&&&&&&&&&&&&&&&&&&&&&&&&&&&&&&&&&&&&&&&&&&&&&&&&&&&&&&&&&&&&&&&&&&&&&&&&&&&&&&&&&&&&&&&&&&&&&&&&&&&&&&&&&&&&&
\section{ Power spectrum in cosmological simulations }
%&&&&&&&&&&&&&&&&&&&&&&&&&&&&&&&&&&&&&&&&&&&&&&&&&&&&&&&&&&&&&&&&&&&&&&&&&&&&&&&&&&&&&&&&&&&&&&&&&&&&&&&&&&&&&&&&&&&&&&

