\chapter{ Cosmological simulations }


In cosmology one issue that currently exists is finding 
the dynamical evolution of the Universe, from its first 
stages to the actual epoch. This makes necessary tools to
study a big amount of interacting particles. 
Considering all possible components in the Universe evolution, 
the particles required to simulate the interaction among them
would be so huge that it would not be viable
to predict the evolution of every single particle for every
time step. 

In cosmological simulations a key component to consider
is dark matter, since it is mostly because of this one that 
the Universe has a filament structure. 
The latter asseveration is due to dark matter dominates the gravitational
interaction, not only because of its amount compared to
baryonic matter but also because it only interacts in this way.

This component cannot be observed directly cause it does not interact 
with radiation. But, since its gravitational effects are strong, there are 
observational evidence that accounts for its existence. For example, 
observing in galaxy rotation curves, the velocity measured at the outskirts 
was too fast to explain only with baryonic matter. 

A key argument to ignore baryonic matter in simulations is that
defining as total density the sum of baryonic and dark matter only, 
dark matter would contribute with around $80\%$ of all the density
content. 

The next chapter is divided in several sections, the first one
corresponds to the methods used in cosmological simulations to
calculate the gravitational evolution of the system. The second
one contains different criteria selection to detect a dark matter
halo in simulations. The third and fourth sections are dedicated
to explain how to build two different statistical measures of clustering
in real and fourier space, the correlation function and power spectrum 
respectively. 


%&&&&&&&&&&&&&&&&&&&&&&&&&&&&&&&&&&&&&&&&&&&&&&&&&&&&&&&&&&&&&&&&&&&&&&&&&&&&&&&&&&&&&&&&&&&&&&&&&&&&&&&&&&&&&&&&&&&&&&
\section{ Numerical methods }
%&&&&&&&&&&&&&&&&&&&&&&&&&&&&&&&&&&&&&&&&&&&&&&&&&&&&&&&&&&&&&&&&&&&&&&&&&&&&&&&&&&&&&&&&&&&&&&&&&&&&&&&&&&&&&&&&&&&&&&




%&&&&&&&&&&&&&&&&&&&&&&&&&&&&&&&&&&&&&&&&&&&&&&&&&&&&&&&&&&&&&&&&&&&&&&&&&&&&&&&&&&&&&&&&&&&&&&&&&&&&&&&&&&&&&&&&&&&&&&
\section{ Halo selection }
%&&&&&&&&&&&&&&&&&&&&&&&&&&&&&&&&&&&&&&&&&&&&&&&&&&&&&&&&&&&&&&&&&&&&&&&&&&&&&&&&&&&&&&&&&&&&&&&&&&&&&&&&&&&&&&&&&&&&&&
%######################################################################################################################
\subsection{ Friends of friends }
%######################################################################################################################
%######################################################################################################################
\subsection{ Bound density maximum }
%######################################################################################################################



%&&&&&&&&&&&&&&&&&&&&&&&&&&&&&&&&&&&&&&&&&&&&&&&&&&&&&&&&&&&&&&&&&&&&&&&&&&&&&&&&&&&&&&&&&&&&&&&&&&&&&&&&&&&&&&&&&&&&&&
\section{ Correlation functions in cosmological simulations }
%&&&&&&&&&&&&&&&&&&&&&&&&&&&&&&&&&&&&&&&&&&&&&&&&&&&&&&&&&&&&&&&&&&&&&&&&&&&&&&&&&&&&&&&&&&&&&&&&&&&&&&&&&&&&&&&&&&&&&&



%&&&&&&&&&&&&&&&&&&&&&&&&&&&&&&&&&&&&&&&&&&&&&&&&&&&&&&&&&&&&&&&&&&&&&&&&&&&&&&&&&&&&&&&&&&&&&&&&&&&&&&&&&&&&&&&&&&&&&&
\section{ Power spectrum in cosmological simulations }
%&&&&&&&&&&&&&&&&&&&&&&&&&&&&&&&&&&&&&&&&&&&&&&&&&&&&&&&&&&&&&&&&&&&&&&&&&&&&&&&&&&&&&&&&&&&&&&&&&&&&&&&&&&&&&&&&&&&&&&

