\chapter{Summary and conclusions}


The bayronic acoustic oscillations is an imprint that appears in the distribution of matter 
for large scales. And it is mainly studied because its size scarcely changes with the evolution of
structure formation. Particularly, we want to find if there is some change in BAO properties with 
the tracer halo population used.
Using a halo catalogue from the MDPL simulation, different populations were 
constructed. 

A first experiment was perfomed for the thick populations shown in \ref{pophalos}.
The CF was found for every population and the BAO signal was extracted . 
Later, using a Gaussian fit the BAO properties were obtained, i.e., amplitude, position 
and width. 
From this first results, the figures \ref{prop1} and \ref{sigm1} were obtained where
the it is shown how every propertie changes with the halo mass considered. 
For the amplitude it can be concluded that it suffers an increase
with halo mass increase. This is due to the bias that is bigger as the halo masses considered
are bigger. 
For the BAO position, it was found among every halo mass considered a difference of around $3\%$.
Though it is a small diference, the result is of the same order of what was found in 
\cite{motion}. 
For the width, there is a change of the width with the halo mass, except for the second 
population. The general trend is that BAO width is samller for more massive halos. As there is an increase in the halo mass of the population studied, it is obtanied
more clustered populations. This could be explained because the dispersion of velocities of
the halos becomes smaller as it is increased the halo mass considered. 
This relation is not satisfied for the population $2$	 hence it appears as a motivation to
perform a smaller binned, specially in this region to study in more detail how the 
amplitude behaves.
	
		
In a second experiment that is esentially equal to the previous one but using the 
thin populations. 
Once again the CF is calculated for every population and the BAO signal is extracted. 
And using a BAO fit the properties of interest are found. Thus, using this results the 
plots \ref{prop2} and \ref{sigm2} are constructed. 

To analyse them, two regions for the plots are used. A linear region, where it is observed
a similar behavior to the ones recovered in the first experiment for each property. This is, 
from the population $10^{12.75}$ $M_{\odot}/h$ to the most massive population, it is observed a similar 
tendency than the one observed in the previous figures. 
The nonlinear region is from masses smaller than $10^{12.75}$ $M_{\odot}/h$ to the less massive ones where
the density modes are coupled among them. 

Now, the amplitude in the left figure of \ref{prop2} recovers a similar shape but
for smaller mass scales, we argue that the nonlinear gravitational effects causes that the tendency
expected is not properly recovered. 	

The width of the BAO plotted in the right figure \ref{prop2} for linear region has the 
same tendency once again. But for nonlinear region, it can not be explained the behavior 
observed and it is out of the scope of this work.

Similarly, the position of the BAO observed in the figure \ref{sigm2} can be analyzed
for the two mentioned regions and in the nonlinear region, we could not recovered the 
same expected tendency. 

Concluding, for populations with smaller masses than $10^{12.75}$ $M_{\odot}/h$ the nonlinear gravitational
effects lead to a non expected behavior that can be not easily explained and that is further
of the interest of this work. But in bigger mass scales, a tendency is recovered for each
property. The main behavior is explained with the velocity field of the structures ``within'' the
BAO. For nonlinear gravitational effects the velocity field causes a shift to bigger scales 
of the BAO peak. This is what is observed in the figure \ref{sigm2} where as the halo mass 
increases the BAO position diminishes. Furthermore, the velocity dispersion of the different
populations depend on the mass scale. Thus, the bigger the halo mass considered the smaller
the velocity dispersion is. Hence, the width of the BAO must decrease as the halo mass taken
increases. This is precisely what is observed in the right figure \ref{prop2} for the linear region. 
Also, the amplitude diminishes while the halo mass increases as shown in the left 
figure \ref{prop2} due to the bias. 
According to the last results, it is concluded that the BAO properties depend on the 
halo mass tracer for the linear region defined. 
