\chapter{Summary and conclusions}


The bayronic acoustic oscillations is an imprint that can be recovered from the distribution of matter 
of large scales, and it is mainly studied because its size scarcely changes with the evolution of
structure formation. Particularly, we want to find if there is some change in BAO properties with 
the tracer halo population used.
For this purpose, a halo catalogue from the MDPL simulation was used.
Thus, it was possible to construct different populations from the halo catalogue.

A first experiment was perfomed for the thick populations shown in \ref{pophalos}.
The CF was found for every population and the BAO signal was extracted . 
Later, using a Gaussian fit the BAO properties were obtained, i.e., amplitude, position 
and width. 
From this first results, the figures \ref{prop1} and \ref{sigm1} were obtained, where
the it is shown how every property changes with the halo mass considered. 
For the amplitude, it can be seen that it increases with the increase of halo mass.
This is due to the bias that is bigger as the halo masses considered are bigger. 

For the BAO position, it was found among every halo mass considered a difference of around $3\%$.
Though it is a small difference, the result is of the same order of magnitude as encountered 
in \cite{motion}. 
For the width, there is a change of it with the halo mass, except for the second 
population. The general trend is that BAO width is smaller for more massive halos. As there is an increase in the halo mass of the population studied, it is obtanied
more clustered populations. This could be explained because of the dispersion of velocities of
the halos, they become smaller as it is increased the mean of the halo masses considered. 
This relation is not satisfied for the population $2$, hence it appears as a motivation to
perform a smaller binned, specially in this region to study in more detail how the 
amplitude behaves.
	
		
A second experiment was performed, esentially equal to the previous one, 
but using the thin populations. 
Again the CF is calculated for every population and the BAO signal is extracted. 
Thus, the properties of interest are found from the BAO fit. Hence, 
using this results, the plots \ref{prop2} and \ref{sigm2} are constructed. 

To analyse them, two regions for the plots are used. A linear region, where it is observed
a similar behavior to the ones recovered in the first experiment for each property. This is, 
from the population $10^{12.75}$ $M_{\odot}/h$ to the most massive population, population 4, it is observed a similar tendency than the one observed in the previous figures. 
The nonlinear region is from masses smaller than $10^{12.75}$ $M_{\odot}/h$ to the less massive 
ones, $q_4$ of population 1, where the density modes are coupled among them. 

Now, the amplitude in the left figure of \ref{prop2} recovers a similar shape but
for smaller mass scales. We argue that the nonlinear gravitational effects causes that the tendency
expected is not properly recovered. It would be expected that amplitude increases with
the halo mass. 

The width of the BAO plotted in the right figure \ref{prop2} for linear region has the 
same tendency once again. But for nonlinear region, there is no pattern recovered, 
no tendency or expected dependence of width with halo mass. To explain this behavior,
one can think that nonlinear effects play an important role. But, since it is not
an easy task by far, we are going to limit to study the linear region, i.e., 
to explain this nonlinear region is out of the scope of this work.

Similarly, the position of the BAO observed in the figure \ref{sigm2} can be analyzed
for the two mentioned regions and in the nonlinear region, we could not recovered the 
same expected tendency. 

Concluding, for populations with smaller masses than $10^{12.75}$ $M_{\odot}/h$ the nonlinear gravitational
effects lead to a non expected behavior that can be not easily explained and that is further
of the interest of this work. But in bigger mass scales, a tendency is recovered for each
property. The main behavior is explained with the velocity field of the structures ``within'' the
BAO. For nonlinear gravitational effects the velocity field causes a shift to bigger scales 
of the BAO peak. This is what is observed in the figure \ref{sigm2}, where as the halo mass 
increases, the BAO position diminishes. At least, this is valid for the linear region. 
Furthermore, the velocity dispersion of the different
populations depend on the mass scale. Thus, the bigger the halo mass considered, the smaller
the velocity dispersion is. Hence, the width of the BAO must decrease as the halo mass taken
increases. This is precisely what is observed in the right figure \ref{prop2} for the linear region. 
Also, the amplitude increases while the halo mass increases as shown in the left 
figure \ref{prop2} due to the bias. 
According to the last results, it is concluded that the BAO properties depend on the 
halo mass tracer for the linear region defined. 

The previous results led us conclude that the BAO position was shift to smaller distances
for larger halo masses, around $\sim 4\%$. This has a direct impact on the parameters of the 
equation of state of the dark energy. As it was mentioned in \cite{motion}, \cite{uno}, \cite{dos}, \cite{tres}, \cite{cuatro}, \cite{crocce}, the importance of the change in the position of the 
BAO peak of $\sim 1\%$ lead to a change of the dark energy parameters of $\sim 4\%$. Thus, with 
the results that we found should imply a notorious change in the dark energy parameters. 

But, let us see this in more detail. There is a relation between the characteristic scale, $s_{||}$, of the BAO, and the Hubble parameter: $H(z=0) \propto s_{||}^{-1} $. 
Hence, a change in the BAO size measured, for instance, a BAO signal recovered from observations,
would lead to a change in the Hubble parameter obtanied. 

Before continuing with our analysis, let us consider the Friedmann equation evaluated for $z=0$:

\[ E(z=0) = \sqrt{\Omega_m+\Omega_{\Lambda}f(z=0,w)+\Omega_k+  \Omega_r},\]


where all the density parameter terms are constant. Thus, a change in the BAO signal that would cause a change in the Hubble parameter, it would also cause a change in the dimensionless dark energy density, $f(z=0,w)$.  The dimensionless dark energy density depends on the  dark energy parameter $\omega$

\[ \omega(z=0) = \omega_0.\]

So, a change in BAO parameter size causes a change in the dark energy parameters found.
For example, dark energy parameters obtained from a LRGs survey, could have a bias in the determination of the dark energy parameter. This is because the LRGs survey would have a specific range mass. 
And as it was shown in this work, the mass of the population would lead to a different BAO signal.
Thus, a specific range mass would lead to a bias estimation of dark matter parameters. 
