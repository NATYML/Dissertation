\chapter{ Results }

In this chapter, the questions initially stated are going to be studied in detail, 
looking forward to answer them. Though, there are some questions, the central one
consists in finding if there is any difference in BAO properties when the scale of the tracer
halo population is changed. A possible way to account for this, it is to study the 
clustering at BAO scales for such halo populations. That is precisely what is proposed
in this work. 
One statistical tool that will be used to study clustering were already explained in chapter \ref{chap3},
the correlation function will provide our main results. 


Now, using this tool, it will be studied the simulation Multidark Planck (MDPL). 
It belongs to the MultiDark database in which Planck parameters were used
(see table \ref{plancktable}). The characteristics of MDPL
simulation are a box length of $L=1$Gpc/h, a number of particles equal to 
$3840^3$ and a mass resolution of $1.51\times 10^9 M_{\odot}/h$.
Furthermore, the MDPL simulation has available a halo catalogue constructed with Friend of 
Friends algorithm and a linking length of 0.2 \footnote{ Data taken from \url{https://www.cosmosim.org/cms/simulations/MDPL/}}. 
It is constructed several populations from this halo catalog, where the mass of the halos is used to
classify them. This is more clear in the table \ref{pophalos} where each mass range for 
the four different populations constructed are shown. They are going to be called the 
thick bins. 

But, the populations have a range mass that is arbitrary, so
the results obtained can have some sort of bias. To avoid this, 
it will be studied in further detail the effect that 
the size of the mass bins have in the results, i.e., the mass
range considered. The bin size could be ``masking'' information
contained for smaller mass ranges. 	

Hence, there are other subsamples created from every thick bin.
Four populations are constructed for the population $1$,  
each of them has the same number of halos. %as shown in figure \textit{pendiente}. 
They are labelled as $q_i$ with $i=1,\dots,4$, where $q_1$ is the population
nearer to population 2.
Likewise, for the population $2$ other four populations are built where
the same labelling is used, where $q_1$ is the population nearer to population 3 
and $q_4$ is the population nearer to population 1. They will be called quarter populations or thin
populations. 
Their mean masses in logarithmic scale are: quartiles for population 1: 
$10^{11.77}$, $10^{11.39}$, $10^{11.19}$ and $10^{11.06}$ $M_{\odot}/h$, quartiles for population 2:
$10^{12.75}$, $10^{12.39}$, $10^{12.19}$ and $10^{12.06}$ $M_{\odot}/h$.
There was another reason to construct such populations, the correlation function
found for the population 2 has an irregular shape around values of BAO peak. 
	
\begin{comment}
Aditionally, other bins called the thin bins are created. The mean mass for each 
bin are $10^{12.5}, 10^{12.9}, 10^{13.3}, 10^{13.7}, 10^{14.1}$ and $10^{14.5}$. 
\end{comment}

Concluding, all of these subdivisions are created to study the effect of the scale of the populations 
on the BAO signal. This is found to be related with the nonlinear gravitational effects, i.e.
for smaller mass halos there is a coupling among different density modes, that is not
so strong for larger scales. 

This lead us to other experiment, to revise if there is a difference in the BAO properties
found for the same cosmological box but for $z=1$, where nonlinear gravitational effects
should be less prominent. In table \ref{z1}, we show the halo populations used 
for $z=1$. The same procedure that will be exposed for the populations with $z=0$ is followed
for $z=1$, but we are going to concentrate in the first redshift where a deeper study 
was performed. 

\begin{table}
\begin{center}
  \begin{tabular}{ | c | c | c | }
    \hline \hline
    MDPL population & Mass range $M_{\odot}/h$& Number of halos \\ \hline \hline
    Pop 4 & $ M \geqslant 1\times 10^{14}$ & 32,436 \\ \hline
    Pop 3 & $ 1\times 10^{13} \leq M < 1\times 10^{14} $ & 443,356\\ \hline
    Pop 2 & $ 1\times 10^{12} \leq M < 1\times 10^{13}$ & 3,687,677\\ \hline
    Pop 1 & $ 1\times 10^{11} \leq M < 1\times 10^{12}$ & 32,868,688 \\ \hline
  \end{tabular}  
   \caption{ Populations constructed from MDPL halo catalogue for z=0. }
\label{pophalos}
\end{center}
\end{table}


\begin{table}
\begin{center}
  \begin{tabular}{ | c | c | c | }
    \hline \hline
    MDPL population & Mean Mass $M_{\odot}/h$& Number of halos \\ \hline \hline    
    9 & $ 1\times 10^{13.5}$ & 8,462\\ \hline
    8 & $ 1\times 10^{13}$ & 66,473 \\ \hline
    7 & $ 1\times 10^{12.5}$ & 326,163 \\ \hline
    6 & $ 1\times 10^{12}$ & 1,258,759 \\ \hline
    5 & $ 1\times 10^{11.5}$ & 4,333,890 \\ \hline
  \end{tabular}  
   \caption{ Populations constructed from MDPL halo catalogue for $z=1$. }
\label{z1}
\end{center}
\end{table}



\section{ Correlation functions for MDPL populations}


\begin{figure}[htbp]
   %\begin{center}
   $
    \begin{array}{cc}
    \includegraphics[width=85mm]{Images/chapter4/CF_all_1e11.pdf}&
    \includegraphics[width=85mm]{Images/chapter4/CF_all_1e12.pdf}\\
   	(a) & (b) \\
    \includegraphics[width=85mm]{Images/chapter4/CF_all_1e13.pdf}&
    \includegraphics[width=85mm]{Images/chapter4/CF_all_1e14.pdf} \\
    	(c) & (d)
    \end{array}$
  %\end{center}
   \caption{ The correlation functions for the populations: (a) $1$ with $N_{random} =10$, (b)population $2$ with $N_{random} =10$, (c) population $3$ with with $N_{random} =10$ and population $4$ with with $N_{random} =40$. For each population 3 runs with 12 bins were performed. }
   \label{CFall}
\end{figure}


To compute the correlation function, it was implemented a parallel C code 
that uses MPI.
There are several important quantities in order to run the code
and obtain the correlation function for a specific population: 
the random sample factor $N_r$, the minimum radial value $R_{min}$, 
the maximum radial value $R_{max}$, the radial number of bins $N_{bins}$,
and the number of particles of the population $N_{part}$.
The random sample factor accounts for the size of the random-random 
catalogue, the total population for the RR catalogue is 
$N_r N_{part}$.
The quanties $R_{min}$ and $R_{max}$ define a range where correlation
function is found. For each range, or equivalently, each shell between 
$R_{min}$ and $R_{max}$, the correlation function is computed
measuring distances among the particles. 
Specifically, it was estimated using the LS estimator 
shown in section \ref{STM}.


In the figure \ref{pophalos}, the correlation function for every population 
is shown. The random factor used is not the same for each case, 
it diminish with the bigger populations since it would be too expensive
computationally to use a big enough value for all of them. For each population, 
three different runs were performed with the same $N_r$ and $N_b = 12$. The first one used 
$R_{min}=20$ and $R_{max}=200$ $Mpc/h$, the second one $R_{min}=25$ and $R_{max}=205$ $Mpc/h$
and the last one $R_{min}=30$ and $R_{max}=210$ $Mpc/h$. In this way the noise 
in the final correlation function plotted per population is reduced. 
This is, if only one run from $R_{min}=20$ to $R_{max}=210$ $Mpc/h$ with bins width of $5$
$Mpc/h$ would have been performed, less data would have been used to calculate the correlation function 
per bin compared with the 3 runs performed, making more robust the results in the last scenario. 

For figures \ref{pophalos}, the correlation function obtained has a similar 
shape in each case. There is a bump around $\sim 105$ $Mpc/h$ that corresponds
to the BAO peak for every correlation function. 
But, for each particular population, there is a difference in the amplitude of the 
correlation function. The more massive halo populations have a larger amplitude
compared with the lower ones. This is more clear from the maximum value obtained in
each case. 

Further, in figure \ref{pophalos}, it is shown an error bar. 
The error was calculated using ten realizations, i.e., it was calculated
the correlation function using the same parameters. But, since the random catalog
changes in every run, the final correlation function obtained is not equal. 
The standard deviation found using the realizations
are the values used for the error bars. 

\subsection{ Random Sampling and Number of particles}


Since the random sampling factor is intended to reduce the shot 
noise, it is important to study its impact on the calculation
of the correlation function (CF). A first exercise in this direction
is displayed in the figure \ref{nrandom} for the population 4, 
the correlation function was calculated for 3 different random sampling factors. 
There is something important to highlight, there were performed
ten realizations per $N_{r}$ and the CFs shown are the mean of these
realizations. From figure \ref{nrandom}, we see that there is no 
significant change in the CF obtanied due to the random sampling number, 
not at least for radial values smaller than $150$ $Mpc/h$. 
For bigger values, the CF with a smaller random sampling factor 
behavior becomes a little bit noisier but this scales are not of our interest.

If the same exercise is repeated with specific realizations instead of the
mean CFs, more fluctuatios are introduced producing a more notorious difference
for scales larger than $150$ $Mpc/h$. 

%**********************************************************************************************************************
\begin{figure}[htbp]
       \centering
               \includegraphics[width=0.7\textwidth]{Images/chapter4/CF_1e14_Nrandom.pdf}
       \caption{\small Correlation function for population 4 with different random factor
       numbers. Using the factor $r^2$ the BAO bump is more notorious.  }
       \label{nrandom}
 \end{figure}
%**********************************************************************************************************************


Other exercise performed to find the effect of the $N_r$ on the CF is shown
in figure \ref{CFscat}. In this case, for a thick population, $10^{14.5}$ $M_{\odot}/h$
several realizations with the same characteristics were performed except
by the $N_r$ used. The two values for $N_r$ are 50 and 100 with equal number
of realizations per calculation. The plotted line in each CF corresponds to the 
mean of the realizations. It was taken a smaller x-range for plots in figure \ref{CFscat},
to see in more detail the bump of the BAO. 
The dispersion of the CFs values for the two $N_r$ values
is very similar. Hence, there is no change in the CF estimation because of 
$N_r$ value.
Though the left figure appears with more points per bin, this is caused
because of the difference in the number of realizations done. 


\begin{figure}[htbp]
   %\begin{center}
   $
    \begin{array}{ll}
    \includegraphics[width=80mm]{Images/chapter4/scatter_mean_1e145_50.pdf}&
    \includegraphics[width=80mm]{Images/chapter4/scatter_mean_1e145_100.pdf}
    \end{array}$
  %\end{center}
   \caption{Correlation function for population $4$ with two $N_r$ values.
   Left: it displays the mean of fifty different CFs. Right: it shows the mean of a one hundred CFs.}
   \label{CFscat}
\end{figure}


There was another factor that has to be considered, the number of particles taken for the 
correlation function calculation. Since there is a large number of particles for the 
population 1 and 2, there is not enough computational resources to run the program, hence
it is necessary to use a subsample representative of all of population. Now, the idea
is to study the effect it has on the estimation of the correlation function. 
In the figure \ref{NofP}, it is shown the correlation function for the population 1 obtained
for two different subsamples. One is around $7.6\%$ of the total population. The other one is around $15.2\%$ percentage of the total population. Both
correlation functions coincide for ranges lower to $80$ $Mpc/h$. In the region where BAO
appears, there is a bigger discrepancy between the two curves. 


%**********************************************************************************************************************
\begin{figure}[htbp]
       \centering
               \includegraphics[width=0.7\textwidth]{Images/chapter4/CF_1e11_NofP.pdf}
       \caption{\small Correlation function for two different subsamples of population 
       population 1 .}
       \label{NofP}
 \end{figure}
%**********************************************************************************************************************



\section{ Correlation function fit }

A first step torward obtaining the BAO signal and its properties from the CF,
it is to make a fit precisely of the CF. In figure \ref{CFall}, the CF 
for different populations are shown, the label used for them is \textit{data}. 
As it can be seen the $y$ axis corresponds to $r^2\xi(r)$, since it allows to visualize better the BAO bump.

The CF function fits, shown in figure \ref{CFall}, were not performed using
all the dots where the CF is known. It was not included the dots that
correspond to the BAO bump. Since the fits try to reproduce the CFs
without including the BAO signal. 

The fit of the points were adjusted to the function, 

\begin{equation}
f(r) = r^2\xi(r) = \frac{r^\lambda}{a_0},
\label{CF_r2}
\end{equation}

\begin{figure}[htbp]
   %\begin{center}
   $
    \begin{array}{ll}
    \includegraphics[width=85mm]{Images/chapter4/CF_all2_1e11.pdf}&
    \includegraphics[width=85mm]{Images/chapter4/CF_all2_1e12.pdf}\\
    (a) &  (b)\\
    \includegraphics[width=85mm]{Images/chapter4/CF_fit_all_1e13.pdf}&
    \includegraphics[width=85mm]{Images/chapter4/CF_fit_all_1e14.pdf} \\
    (c) &  (d)
    \end{array}$
  %\end{center}
   \caption{The correlation functions and fits for the populations (a): $1$, (b): $2$, (c): $3$ and (d): $4$ 
   are displayed from the upper left to the lower right. For each population three runs with 12 bins
   were carried out. }
   \label{CFall}
\end{figure}


where $\lambda$ and $a_0$ are the parameters found with the fit. 
The theoretical function used to adjust $\xi(r)$ was already shown in \ref{powlaw}. 
Since it is is a power law function, an easier way to carry out the fit is in 
logarithmic scale. In this way, it is performed a linear fit: $a r +b$, 
but, the values of the CF can be negative. Then, to avoid any trouble a $\Delta$ value 
is summed to the CF to get only positive values before perfoming the fit. 

When the coefficients $a$ and $b$ are obtained, we can recover the initial parameters
$\lambda$ and $a_0$  by considering the fact that $\gamma = a$ and $a_0=\exp(-b)$. 
Now, replacing the parameters in the expression \ref{CF_r2} and subtracting the amount $\Delta$, 
the CF can be plotted. 

This procedure is repeated to obtain the fit for every population and 
all of its realizations. In the figure \ref{CFall} some of the fits 
obtained for the samples are displayed. Here, 
one important thing is to get a measure of the robustness of the fit. Hence, 
error bars are calculated using the different realizations of the CF, i.e.,
the standard deviation is obtained. 
The error bars are shown for every figure of \ref{CFall} but 
because of the similarities of the fits they are almost no visible. 
So the CF fits are considered robust enough.  

In all of the CFs in \ref{CFall} there are visible two bumps, the first one
corresponds to the BAO since it agrees with the value observed of BAO
peak as it was shown in figure \ref{ps_cf}.
Also the position coincides with the BAO position measured for galaxy clusters, 
a mass range we are considering in our populations. 

Something to highlight is that in the population $2$, 
there is no separation of these two bumps, making more difficult to recognize the BAO signal. 
This was precisely the reason to study in more detail the effect of the mass bins in the CF estimation 
and thus the BAO bump. 


\subsection{BAO fit}

Since the measure of our interest is the BAO bump, it becomes necessary
to extract it from the CF and thus to be able to obtain the properties of 
BAO we are looking for to analyse. In this direction, a correlation function 
model can be useful,  

\begin{equation}
CF(r) = \xi(r) - \Delta + GF(r,A,\mu,\sigma),
\end{equation}

the term $\xi(r)$ corresponds to the theoretical form shown in equation \ref{powlaw}.
Numerically, we could calculate the correlation function from the CF fit divided by $r^2$. 
Since the CF fit was $r^2\xi(r)$. 
The second term is the one that it is summed to the CF as was explained
in the previous section, so it must be subtracted here. 
The function $GF$ is a gaussian fit that reproduces the BAO shape recovered 
from the correlation function. 

\[GF(r,A,\mu,\sigma) = Ae^{-(r-\mu)^2/(2\sigma^2)}.\]

This model, where BAO is fitted with a gaussian function is proposed in \cite{motion}. 

Then, the next steps are followed to recover the BAO bump for every population. 
The median of the realizations is taken as the main CF and the standard
deviation is obtained through the realizations.

\begin{itemize}

\item[1)] The term $\Delta$ is substracted from the CF fit. After the function
is divided by $r^2$ obtaining $\xi(r)$. 

\item[2)] The function $\xi(r)$ is substracted from the estimated CF leaving
the signal of the two bumps. 

\item[3)] Only the points corresponding to the first bump located around $105$ $Mpc/h$
are selected. 

\item[4)] A gaussian fit of the first bump is performed. Only the more central points
of the signal are considered since the outer ones are the noiser parts of the signal. This noise 
could be diminished using more realizations per population. 

\item[5)] The fit is also performed for all the realizations, the same points
considered for the BAO signal fit of the mean CF are taken for the remaining realizations. 

\item[6)] The parameters amplitude $A$, mean $\mu$ and standard deviation $\sigma$ are
obtained for every population and the realizations. 

\end{itemize}

The parameters that characterizes the BAO bump are the amplitude $A$, the position
$\mu$ and the width $2\sigma$. In the figure \ref{BAO_fit} some fits of the BAO
signal are shown. It can be noticed that the points that correspond to the BAO
bump have a gaussian-like distribution. The Gaussian fits performed are also shown
with the parameters found, blue curve. It can be noticed a good coincidence between 
the data (green curve) and the fit obtained. 
The error bars displayed were obtained through the BAO fits performed for the
different realizations, with them the standard deviation per population was found.  


\begin{figure}[htbp]
   %\begin{center}
   $
    \begin{array}{cc}
    \includegraphics[width=85mm]{Images/chapter4/BAO_1e14.pdf}&
    \includegraphics[width=85mm]{Images/chapter4/BAO_1e13.pdf}\\
    (a) &  (b)\\
    \includegraphics[width=85mm]{Images/chapter4/BAO_1e12_q4.pdf}&
    \includegraphics[width=85mm]{Images/chapter4/BAO_1e11_q1.pdf} \\
    (c) &  (d)
    \end{array}$
  %\end{center}
   \caption{ BAO signal for population (green curve) (a): 4, (b): 3, (c): $q_4$ of $1e12$ and (d): $q_1$ of $1e11$. Two fits are shown, magenta curve: Basis spline fit and blue curve: Gaussian fit. 
}
   \label{BAO_fit}
\end{figure}

For every plot displayed in the figure \ref{BAO_fit}, there is a curve labeled as B spline,
magenta curve.
This fit was obtained using basis splines. It has a different behavior than the one observed 
for the Gaussian fit. At least in general terms, it fits better the BAO signal recovered.
Furthermore, there is a difference in the main peaks of the two fits 
performed, Gaussian and basis spline fit. Despite the difference between the peaks, the
functional form obtained for the properties is similar. This will be shown in the next section. 

\section{ BAO properties in the populations of MDPL }

As mentioned previously, the BAO peak is clearly detected for every population 
and the fit to the BAO signal was properly calculated using a Gaussian function. 
Let us see the situation in more detail. There are different populations, each of 
them have the ranges in mass shown in the beginning of this chapter (see table \ref{pophalos} and table 
\ref{z1} and the quartil populations).
  
In the case of thin populations, $q_1$, $q_2$, $q_3$ and $q_4$ of the populations $1$ and 
$2$, their mean mass increases from $q_4$ to $q_1$. 
This means that the halo masses considered
in population $q_4$ are smaller than the halo masses considered in $q_1$. 
That way, we are be able to analyze if there is an effect on the BAO properties obtained for each population. 
It is important to take into account that more massive halos trace higher density 
peaks in the matter density field. This should lead, in principle, to a stronger 
correlation in the most massive populations compared with the less massive ones. 
Thus, a better detection of the BAO signal. In the left figure \ref{prop1}, an increase of the amplitude of the BAO for more massive halos is obtained as precisely expected for a stronger 
correlation. 

%\huge BAO amplitude $Mpc$ $h^{-1}$

\begin{figure}[h]
   %\begin{center}
   $
    \begin{array}{cc}
    \includegraphics[width=78mm]{Images/chapter4/Amp_vs_Mh.pdf}&
    \includegraphics[width=78mm]{Images/chapter4/Pos_vs_Mh.pdf}
    \end{array}$
  %\end{center}
   \caption{\small In the left panel the amplitude of BAO in function of the mass of the thick populations 
   is shown. In the right one the position of BAO in function of the mass of the thick population is displayed.}
   \label{prop1}
\end{figure}

The initial position of the BAO depends on the sound horizon scale as mentioned in \ref{SH},
but as explained in the model exposed in \ref{shiftBAO}, this position changes due to
nonlinear effects. Now, left figure of \ref{prop1} shows the position of the 
BAO in function of the halo mass. It can be noticed a decrease in the value of the position
as the halo mass increase. This behavior could be expected due to nonlinear gravitational 
collapse. The velocity field causes a movement of the BAO peak to bigger scales 
for the nonlinear regime. In this case, this corresponds to the 
smaller halo masses for which the density modes are coupled among them. 
The porcentual difference is around $3.8\%$. This is
precisely of the same order of magnitude found in \cite{motion}. This result contributes 
in a different way as \cite{motion}, since they consider cosmological boxes with a very
small resolution $640^3$ in contrast to the populations used in this study. Furthermore,
we study the BAO behavior for a more extended mass range and this could let us see 
the effect of BAO properties in the nonlinear range. 

%**********************************************************************************************************************
\begin{figure}[htbp]
       \centering

    \includegraphics[width=90mm]{Images/chapter4/Width_vs_Mh.pdf}
\caption{\small Width of BAO versus the masss of the thick populations.}
       \label{sigm1}
 \end{figure}
%**********************************************************************************************************************
\begin{comment}
But not only the velocity field causes a shift of the BAO peak due to nonlinear evolution, 
there is a change of the width of the BAO peak. In order to understand this, let us consider
that the density field is affected by the dispersion of velocities and thus the correlation 
function obtanied. In this scenario, when we are considering halo populations with smaller 
masses, it is expected a bigger dispersion of the velocities compared with the populations 
with bigger masses. 	This occurs due to more massive halos are harder to move, thus the 
deviation from the mean velocity is not so big. At least,  compared  with the smaller halos 
that can have a wider range of velocities. 
\end{comment}

In order to understand our results, lets propose a toy model that may help 
to see what is happening on the BAO.
Consider a single density perturbation that is sourronded 
by a BAO as it is shown in the figure of \ref{toy_pro} $a$. 
Here, we are going to analyse the behavior of the BAO properties due to nonlinear 
effects. As the system evolves, the velocity field causes changes in the BAO properties. 
To analyse what happens, we make a separation by halo mass and analyse the difference
in the velocity field in each case. 
Now, consider the right figure of \ref{toy_pro} where a toy density profile is shown 
and separated in two parts. The left curve is 
the one halo term  and the right curve is the BAO signal. 
The one halo term is, esentially, the density profile of a halo 
and it is affected by the dispersion of velocities. 
This dispersion causes a broadening of the profile. Hence, a similar behavior 
is expected for the BAO signal. This is, the halos 
in the BAO signal have a dispersion of velocities that causes a broadening.
But, the halo mass considered changes the broadening. 
In this scenario, when we are considering halo populations with smaller 
masses, it is expected a bigger dispersion of the velocities compared with the populations 
with bigger masses. 	This occurs due to more massive halos are harder to move, 
i.e., they move closer to the ordered linear flows. Thus the 
deviation from the mean velocity is not so big. At least,  compared  with the smaller halos 
that can have a wider range of velocities. This idea is supported by the distribution
of velocities observed in different populations as shown in the figure \ref{hist}, where
we see that for larger halo masses populations the velocity dispersion is smaller. This is
in accordance with the toy model proposed. 
In the figure \ref{hist} is also seen a bigger mean velocity for larger masses populations. 
%\huge BAO amplitude $[Mpc$ $h^{-1}]$ 
These velocity distributions were plotted using the halo velocities, taken from the MDPL halo catalog. Thus, a histogram allows to obtain an estimative of the mean velocity per population. 


The previous discussion is a possible explanation of what it is seen in the
figure \ref{sigm1} where the width of the BAO decreases with more massive halos.
But, the behavior of the population $2$ does not follow this tendency. Because of this,
the quarter and thin populations were created to study in more detail the effect of
the mass bins in the properties measure of BAO. 



%\huge v[$Kms^{-1}/h$]

%**********************************************************************************************************************
\begin{figure}[htbp]
       \centering
    \includegraphics[width=100mm]{Images/chapter4/toy_pro.png}
\caption{\small a) Scheme of a BAO. b)Profile that includes one halo term and the BAO signal.}
       \label{toy_pro}
 \end{figure}
%**********************************************************************************************************************

\begin{figure}[h]
   %\begin{center}
   $
    \begin{array}{cc}
    \includegraphics[width=80mm]{Images/chapter4/AmplitudevsMh2.pdf}&
    \includegraphics[width=78mm]{Images/chapter4/WidthvsMh2.pdf}
    \end{array}$
  %\end{center}
   \caption{\small In the left panel the amplitude of BAO in function of the mass of all populations 
   is shown. In the right one width of BAO versus the masss of the thick populations.
   The vertical line in each plot indicates two regions, the left one, the nonlinear region
   and the right ond, the linear region.   }
   \label{prop2}
\end{figure}

%**********************************************************************************************************************
\begin{figure}[htbp]
       \centering

    \includegraphics[width=100mm]{Images/chapter4/PosvsMh2.pdf}
\caption{\small The position of BAO in function of the mass of the thick population is displayed.
   The vertical line indicates two regions, the left one, the nonlinear region
   and the right one, the linear region. }
       \label{sigm2}
 \end{figure}
%**********************************************************************************************************************


\begin{figure}[h]
   %\begin{center}
   $
    \begin{array}{cc}
    \includegraphics[width=78mm]{Images/chapter4/MDPL_Haloes_Vel_10e11.pdf}&
    \includegraphics[width=78mm]{Images/chapter4/MDPL_Haloes_Vel_10e13.pdf}\\
    (a) &  (b)\\
    \includegraphics[width=78mm]{Images/chapter4/MDPL_Haloes_Vel_10e14.pdf}&
    \includegraphics[width=78mm]{Images/chapter4/all_hist.pdf} \\
	(c) & (d)
    \end{array}$
  %\end{center}
   \caption{\small 
   Figure with population $1\times 10^{11}$ has a mean of $519.51$ $km/sh$ and it's estimated variance is $67214$ $km/sh$. 
   Figure with population $1\times 10^{13}$ has a mean of $501.81$ $km/sh$ and it's estimated variance is $58720$ $km/sh$. 
   Figure with population $1\times 10^{14}$ has a mean of $490.21$ $km/sh$ and it's estimated variance is $53063$ $km/sh$. 
   It can be seen that the population $1e14$ has the smaller variance as it is predicted by our model. 
   In the last figure are compared all the populations. 
   }
   \label{hist}
\end{figure}



\begin{comment}
Another important factor is the gravitational interaction among more massive halos.
Since larger masses exert greater gravitational attraction than smaller masses, 
the more massive halos should be closer among them, causing a smaller broadening
in the BAO peak compared with the less massive ones. This can be seen in the
figure \ref{sigm1} where the width of the BAO decreases with more massive halos.
But, the behavior of the population $2$ does not follow this tendency. Because of this,
the quarter and thin populations were created to study in more detail the effect of
the mass bins in the properties measure of BAO. 
\end{comment}

The BAO properties obtained for all the populations, except $1\times 10^{11}$ and $1\times 10^{12}$,
are plotted in the figures \ref{prop2} and \ref{sigm2}. It is noticed that when it is used
thiner masses bins, we do not recover the same tendency observed in the previous
figures. Let us start with the left figure of \ref{prop2}. In this case, there is
still a tendency of the amplitude to increase with an increase of the halo mass.
Though, for masses smaller than $\sim 10^{12.4}$ $M_{\odot}/h$, there is a more irregular behavior
compared with the one obtained for the thicker bins. 

Now, in the right figure of \ref{prop2}, blue curve, it is not recovered the same behavior as \ref{sigm1}.
But, for halo masses bigger than $\sim 10^{12.4}$ $M_{\odot}/h$, the width of the BAO decreases
with mass. This was expected as explained for \ref{sigm1}. For masses smaller than
$\sim 10^{12.4}$ $M_{\odot}/h$, the nonlinear effects causes that there is no tendency of the 
data. The red curve shown in the same figure corresponds to the width in function of the 
halo mass for $z=1$. It has a better behavior since nonlinear effects are not so strong.
Hence, it can be noticed that because of the structure evolution between $z=1$ and $z=0$, 
the coupling among different density modes becomes more notorius. 

Figure \ref{sigm2}, shows the position of the BAO vs the halo mass.
In the case of the solid blue line, the populations considered are those ones that
belong to $z=0$, with the properties recovered using the Gaussian fit. To analyse it, 
let us again divide the figure in two regions. For halo masses bigger than $\sim 10^{12.4}$ $M_{\odot}/h$,
a similar behavior compared with \ref{prop1}, i.e., a decrease in the BAO position with
the increase of the halo mass. As was previously mentioned, this behavior is expected 
as explained in \ref{shiftBAO}. But, for masses smaller than $\sim 10^{12.4}$ $M_{\odot}/h$, it is not 
observed any specific tendency. It is in this region where the coupling among density
modes makes the nonlinear effects too big to extract any possible tendency. Hence, it 
is out of the purpose of this work to go deeper into the subject. 
A second curve is the dashed blue line, it corresponds the same populations 
as the solid blue line. But, in this case, the position of the BAO was recovered
using a basis spline fit. This was perfomed to make the results obtanied
with the Gaussian BAO fit more robust. Although, there is a difference in the properties
recovered, this values are similar and also they behave in a similar way. Thus,
the region for masses larger than $\sim 10^{12.4}$ $M_{\odot}/h$ shows a shift of the BAO position
to smaller distances.
Hence, this supports the result obtained with the Gaussian fit. 
The red solid curve shows the BAO properties recovered for the populations at $z=1$.
It can be seen that BAO position does not change significantly because the nonlinear 
effects are less prominent in this epoch. The last dot of this curve was produced with 
a population were small, hence this specific result is not so reliable. 
Similarly, the dashed red line displays
the properties found for the same populations as the solid red line, but it recovers
the position using a basis spline fit. For both curves, there is not a notorious change 
in the position of the BAO. Hence, these two curves support the idea that nonlinear 
effects are causing a shift in the BAO position, not so evident for $z=1$. 

The curves obtained for $z=1$, in figures \ref{sigm2} and right plot of \ref{prop2},
were constructed in the same way, as curves for $z=0$ were calculated. Thus, the CF
for different populations (see table \ref{z1}) was calculated and its variance. After
recovering the BAO signal from the CF, the fit using a Gaussian function allowed to
calculate the BAO properties at this redshift. 


Concluding, the properties recovered for the BAO using the thicker bins are behaving
as expected but they are possibly "masking" information of smaller mass scales. For this, 
the thiner bins provided us with a more accurate properties. This lead us to see that
for masses larger than $\sim 10^{12.4}$ $M_{\odot}/h$, we recovered the expected tendency. But for
the smaller masses, the nonlinear effects do not let us extract any specific behavior
of the properties. A nonlinear behavior is dominating the BAO properties at these mass
scales. 

\



During this chapter the results for two different numerical experiments were performed
using a halo catologue from MDPL simulation. The first case used thick bins for the populations
built. It was noticed that BAO properties changed due to the scale of the tracer halo population
used. The amplitude increases with larger halo masses due to the bias. The position 
and width decreases for larger halo masses. The last two results were explained using a toy
model that only considered the one halo term and a single BAO contribution for a density
profile as shown in the figure \ref{toy_pro}. 
The second experiment was performed in the same way that the previous one but we built
populations using thin bins. A similar behavior was recovered for BAO properties but 
only for masses larger than $\sim 10^{12.4}$ $M_{\odot}/h$. This lead us to study the
BAO properties in two different regions. For larger masses than $\sim 10^{12.4}$ $M_{\odot}/h$
were recovered a similar behavior compared with the first experiment. But for smaller masses
than $\sim 10^{12.4}$ $M_{\odot}/h$, it was not obtained a specific tendency. We suppose
this behavior is caused by stronger nonlinear effects. 