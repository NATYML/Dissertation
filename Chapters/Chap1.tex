\chapter{Introduction}\label{intro}


In the standard model of cosmology the universe was born in a big bang, a primordial singularity with very
high energy and matter density that finally produced an
expanding, isotropic and homogeneous Universe. From observations it has been found that this expansion  
is currently accelerating with time \cite{Hamuy}.

There are four components of the matter-energy content of the universe, 
dark and baryonic matter, radiation and dark energy. According to recent estimations, the last one accounts for around $70\%$ of this content and is responsible for the accelerated expansion of the universe \cite{Pilar}. 
The baryonic acoustic oscillations allows to study the nature of this expansion as it will be explained.

\

In the early universe the dark matter (DM) formed density fluctuations, causing baryonic matter to be unstable against
gravitational perturbations. At this stage in the evolution of the universe the temperature was very high, allowing a coupling between
baryonic matter and radiation through Thomson scattering. 
So the increase of baryonic matter in the DM density fluctuations not only caused an increase 
in density, but also radiation pressure against collapse. Therefore, an expanding wave centered in the fluctuation
is caused because of the radiation pressure. This wave is the baryonic acoustic oscillation hereinafter BAO \cite{Hu}, \cite{SH02}. 

\	

Nevertheless, it is necessary to consider that the universe is expanding and this results in a  
temperature decrease. Therefore, when temperature is low enough, the baryonic matter and radiation 
decoupled, making BAO to stop expanding and leaving an imprint in the matter distribution. This is, 
a peak in the matter distribution that can be noticed in the correlation function. 
The distance that a BAO could have traveled by the time of
decoupling is called sound horizon. This scale has been measured in the Cosmic Microwave Background 
as $146.8\pm 1.8 \mathrm{Mpc}$, \cite{SIZE}.  

\

Since BAO do not change in size after decoupling they can be used as a standard ruler.  They allow to
measure the Hubble parameter and angular diameter distance as a function of $z$, and this way to measure the rate of expansion at different times during the evolution of the universe. Hence, BAO is key to constraint dark energy parameters. 

\	

A way to study the imprint in the matter distribution associated  to BAO signal is through the 2D point correlation function or the power spectrum that is its Fourier pair, \cite{PLOT}, \cite{PLOT2}.  
A peak due to the BAO appears in the correlation function but there are 
several issues to take into consideration.
For example, the non-linear clustering smear out the BAO imprint causing a broadening of the peak \cite{Shift01}. These, among other problems, have to be taken into account when BAO are studied. 

\

In the present work, we plan to study the BAO from numerical cosmological simulations. 
More precisely, the BAO will be studied trying to answer what are the changes of 
the BAO's properties with the change of the scale of the tracer halo population?
is there any change in the position peak? is there any change in the width peak? 
In general, the question we want to answer is: Is there any dependence in the width 
and amplitude of the BAO signal with the tracer halo population?
Answering this questions will lead not only to a better understanding of the physics of
BAO but a better understanding of the accelerated expansion of the universe that still has so many questions to be answered. 


