\chapter{Introduction}\label{intro}


In the standard model of cosmology the universe was born in a big bang, an explosion that produced an
expanding, isotropic and homogeneous Universe. From observations it has been found that this expansion  
is currently accelerating with time (Hamuy et al.,1996). 

There are several components of the matter-energy content of the universe, 
dark and baryonic matter, radiation and dark energy. According to recent estimations \citep{DEF}, the last one accounts for 
around $70\%$ of this content and is responsible for the accelerated expansion of the universe. 
The baryonic acoustic oscillations allows to study the nature of this expansion as it will be explained.

\

In the early universe the dark matter (DM) formed density fluctuations, causing baryonic matter to be unstable against
gravitational perturbations. At this stage in the evolution of the universe the temperature was very high, allowing a coupling between
baryonic matter and radiation through Thomson scattering. 
So the increase of baryonic matter in the DM density fluctuations not only caused an increase 
in density, but also radiation pressure against collapse. Therefore, an expanding wave centered in the fluctuation
is caused because of the radiation pressure. This wave is the baryonic acoustic oscillation (BAO) (Hu and Sugiyama, 1996;
Eisenstein and Hu, 1998). 

\	

Nevertheless, it is necessary to consider that the universe is expanding and this results in a  
temperature decrease. Therefore, when temperature is low enough the baryonic matter and radiation 
decoupled, making BAO to stop expanding and leaving an imprint in the matter distribution. 
The distance that a BAO could have travelled by the time of
decoupling is called sound horizon. This scale has been measured in the Cosmic Microwave Background 
as $146.8\pm 1.8 \mathrm{Mpc}$, (\cite{SIZE}).  

\

Since BAO do not change in size after decoupling they can be used as a standard ruler.  They allow to
measure the Hubble parameter and angular diameter distance as a function of $z$, and this way to measure the rate 
of expansion at different times during the evolution of the universe. Hence, BAO is key to constraint dark energy 
parameters. 

\	

A way to observe the imprint let by BAO is through the 2D point correlation function or the power spectrum that is 
its fourier pair, (\cite{PLOT}, \cite{PLOT2}).  A peak due to the BAO appears in the correlation function (see figure \ref{ps_cf}) but there are 
several issues to take into consideration.
There is a bias between baryonic and dark matter distribution (\cite{Biases}) and hence in their correlation functions. This bias  
plays an important role when observational data 
is being studied. A method proposed in such cases is suggested in (\cite{HBM}). 
Moreover, the non-linear clustering smear out the BAO imprint causing a broadening of the peak (Crocce
and Scoccimarro, 2008). These, among other 
problems, have to be taken into account when BAO are studied. 

\

Observational studies of baryonic acoustic oscillations have been done in several previous works such 
as \cite{Obs01}, \cite{Obs02}, \cite{Obs03}, \cite{Obs04} . Measurements of baryonic acoustic oscillations on simulations 
have also been done in these works by \cite{Sim01}, \cite{Sim02}, \cite{Sim03}, \cite{Sim04}.
And theoretical studies of baryonic acoustic oscillation using non linear theory have been realized in \cite{Theo01}, \cite{Theo02},
\cite{Theo03}, \cite{last} .  

\

In the present work, we plan to do a comparison between the power spectrum estimated from numerical cosmological simulations
and the one obtained from observations of the Sloan Digital Sky Survey (SDSS). The method to construct the power spectrum
is shown in section \ref{subsubsec:CPS}.2. The method to obtain the dark matter density field for observations is 
explained in section \ref{subsubsec:HBM}.3. In both cases, observations and numerical cosmological simulations, the BAO 
peak will be studied, but what are the changes of the BAO's properties with changing the scale of the tracer halo population?
is there any change in the position peak? is there any change in the width peak? or, is there a damping in the oscillations
caused by BAO in the power spectrum? In general, the question we want to answer is: Is there any dependence in the width 
and amplitude of the BAO signal with the tracer halo population?
Answering this questions will lead not only to profound understanding of the physics of
BAO but a better understanding of the accelerated expansion of the universe that still has so many questions to be answered. 


\section{ Baryonic acoustic oscillations } 



